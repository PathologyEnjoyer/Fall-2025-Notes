
\documentclass[x11names,reqno,14pt]{extarticle}
\input{preamble}
\usepackage[document]{ragged2e}
\usepackage{epsfig}
\usepackage{dynkin-diagrams}

\pagestyle{fancy}{
	\fancyhead[L]{Fall 2024}
	\fancyhead[C]{MAT1344F}
	\fancyhead[R]{John White}
  
  \fancyfoot[R]{\footnotesize Page \thepage \ of \pageref{LastPage}}
	\fancyfoot[C]{}
	}
\fancypagestyle{firststyle}{
     \fancyhead[L]{}
     \fancyhead[R]{}
     \fancyhead[C]{}
     \renewcommand{\headrulewidth}{0pt}
	\fancyfoot[R]{\footnotesize Page \thepage \ of \pageref{LastPage}}
}
\newcommand{\pmat}[4]{\begin{pmatrix} #1 & #2 \\ #3 & #4 \end{pmatrix}}
\newcommand{\A}{\mathbb{A}}
\newcommand{\B}{\mathbb{B}}
\newcommand{\fin}{``\in"}
\newcommand{\mk}[1]{\mathfrak{#1}}
\newcommand{\g}{\mk{g}}
\newcommand{\h}{\mk{h}}
\newcommand{\J}{\mc{J}}
\newcommand{\tphi}{\tilde{\phi}}
\newcommand{\pois}[2]{\{#1,#2\}}
\newcommand{\fibrate}[3]{\begin{tikzcd} #1 \ar[d, "#2"] \\ #3 \end{tikzcd}}
\renewcommand{\t}{\mk{t}}
\DeclareMathOperator{\Perm}{Perm}
\DeclareMathOperator{\pdim}{pdim}
\DeclareMathOperator{\gldim}{gldim}
\DeclareMathOperator{\lgldim}{lgldim}
\DeclareMathOperator{\rgldim}{rgldim}
\DeclareMathOperator{\idim}{idim}
\DeclareMathOperator{\SU}{SU}
\DeclareMathOperator{\SO}{SO}
\DeclareMathOperator{\Ad}{Ad}
\DeclareMathOperator{\ad}{ad}
\DeclareMathOperator{\gr}{gr}
\DeclareMathOperator{\Sig}{Sig}
\newcommand{\Rmod}{R-\text{mod}}
\newcommand{\RMod}{R-\text{Mod}}
\newcommand{\onto}{\twoheadrightarrow}
\newcommand{\into}{\hookrightarrow}
\newcommand{\barf}{\bar{f}}
\newcommand{\dd}[2]{\frac{d#1}{d#2}}
\newcommand{\pp}[2]{\frac{\partial #1}{\partial #2}}
\newcommand{\gl}{\mk{g}\mk{l}}
\newcommand{\spew}{\Sp(E,\omega)}
\newcommand{\jew}{\mc{J}(E,\omega)}
\newcommand{\Specmax}{\operatorname{Spec}_{\operatorname{max}}}
\renewcommand{\P}{\mathbb{P}}
\renewcommand{\E}{\mathbb{E}}
\DeclareMathOperator{\Ext}{Ext}
\DeclareMathOperator{\Rank}{Rank}
\DeclareMathOperator{\Sp}{Sp}
\DeclareMathOperator{\ann}{ann}
\DeclareMathOperator{\Lag}{Lag}
\DeclareMathOperator{\Riem}{Riem}
\DeclareMathOperator{\Span}{span}
\newcommand{\exactlon}[5]{
		\begin{tikzcd}
			0\ar[r]&#1\ar[r,"#2"]& #3 \ar[r,"#4"]& #5 \ar[r]&0
		\end{tikzcd}
}

\title{MAT 1190}
\author{John White}
\date{Fall 2025}


\begin{document}

\section*{Lecture 1, 3/9/25 (Happy birthday to me)}

Oh dear, we're starting with chapter 2 of Hartshorne...

Read chapter 1.1 of Hartshorne before friday

Test your understanding of the important bits against Exercise 1.4(Zariski vs product topology)

Following theorem is perhaps unconventional for an ag class.

We use the 	``Bourbaki conventions:"

\defn

A topological space $X$ is said to be \underline{quasicompact} if for every open cover $X = \bigcup_{i\in I}U_i$, there existss a finite subcover $I' \subset I$ such that $X = \bigcup_{i\in I'}U_i$. 

This is usally called ``compact".

\defn

A topological space is said to be \underline{compact} if it is quasicompact and Hausdorff.

\paragraph{Recall:} $X$ is called \underline{Hausdorff} if for all pairs $(x, y)$ of \underline{distinct} points there exist neighborhoods $U_x, U_y$ of $x, y$, such that $U_x \cap U_y = \varnothing$. 

In French, one uses the term ``separated space."

These terms will reappear in algebraic geometry when studying separated schemes.

\paragraph{}

This property is equivalent to the following: For $(x, y) \in X\times X \setminus \bigtriangleup$ (the diagonal elements $\{(x,x) \mid x \in X\}$), there are neighborhoods $U \ni x, V \ni y$, with $U\times V \cap \bigtriangleup = \varnothing$. 

Hence $U\times V$ lies entirely within the complement of the diagonal. So $(x, y)$ is in the interior of $X\times X \setminus \bigtriangleup$. 

Using the definition of the product topology, one can show that $X$ is a Hausdorff space if and only if $\bigtriangleup$ is closed in the product topology. This is the formulation which will be meaningful when we transport to algebraic geometry. 

\thm[Gelfond-Naymark]

Roughly: 

A compact (quasicompact + hausdorff) topological space can be ``recovered" from the ring $C(X) \eqdef C(X,\R)$ of continuous real-valued functions.

\proof

This is a special case of what they proved. Will get into proof later

\qed

In particular, we want it to be true that if $X, Y$ are two compact spaces with abstractly isomorphic rings of functions, i.e. $C(X) \cong C(Y)$, then $X, Y$ should be homeomorphic, $X \cong Y$. 

\paragraph{\underline{From rings to spaces}}

To fix conventions: 

\defn

When we write ``ring", we always mean a commutative, unital ring. So $C(X)$ is indeed always a ring (obviously). 
\paragraph{}

\underline{First step:}

Try to recover the underlying set of points. 

Ideals: given $x \in X$, we obtain a ring homomorphism, called the \underline{evaluation index at $x$}, $e_x:C(X)\to \R$ which takes a continuous real-valued function and evaluates it at $x$: $f\mapsto f(x)$. 

Since $(f+g)(x) = f(x) + g(x)$, and similarly for multiplication, this really is a ring homomorphism. 

\paragraph{Fact:} This map $(e_x)$ is surjective because of constant functions. 

Thus we have the isomorphisms $\R \cong \frac{C(X)}{\ker(e_x)}$. We refer to the denominator as $\mathfrak{M}_x$, the ideal of functions vanishing at $x \in X$. Note that the quotient is a field, so $\mathfrak{M}_x$ is maximal. 

\defn

Let $R$ be a ring. We denote by $\Specmax(R)$ the set of maximal ideals in $R$. 

\prop

Let $X$ be compact. Then there exists a bijection of sets $X \cong \Specmax C(X)$. 

The precise claim may be summarized as follows: 
\begin{itemize}

\item Every maximal ideal $I$ of $C(X)$ is of the form $I = e_x$ for some $x \in X$. 

\item If $x, y$ are points in $X$, and $\mathfrak{M}_x  =\mathfrak{M}_y$, then $x = y$. 

\end{itemize}

\proof
\qed

What about the topology? Let $R$ be an abstract ring with the additional property that for every maximal ideal $\mk{M}\in\Specmax R$, the quotient $R/\mk{M} \cong \R$. Then we can make the following construction: for every element of the ring, we can associate to every element $f \in R$ a function $f:\Specmax R \to \R$ in the following way: $\mk{M}\mapsto \barf \in \R \cong R/\mk{M}$. 

\paragraph{Aside:}

To an algebraist, we think of $\R|(\bar{\Q}\cap\R)$ as a transcendental extension, $\R = (\bar{\Q}\cap \R)(\alpha_0,\alpha_1,\dots)$. So, there are lots of field automorphisms on $\R$, none of which are continuous. 

Aside over. 

\paragraph{Now:} Look at the coarsest topology on $\Specmax R$ such that all functions $\mk{M} \mapsto f + \mk{M} \in \R$ are continuous for each $f \in R$. 

That is, the topology on $\Specmax R$ is generated by preimages $f^{-1}(U)$, where $f:\Specmax R\to\R$ denotes the map associated with $f \in R$. 

Due to the existence of noncontinuous elements of $\Aut(\R)$, it is problematic to work with the standard topology. 

It is in some way ``unnatural" to think of the topology of $\R$ analyticaly, if we want to do algebra. 

\paragraph{Instead:} We use the cofinite topology on $\R$ instead, i.e. the nonempty open subsets are the complements of finite sets. 

\defn

Let $R$ be a ring. Then the \underline{Zariski topology} on $\Specmax R$ is the topology generated by ``standard open subsets," which are defined as subsets of the form 
\[
U_f = \{\mk{M} \in \Specmax R \mid f\not\in \mk{M}\}
\]

It is in a certain way ``algebraically robust". 

\paragraph{Remark:} The condition that $f \not \in \mk{M}$ has a very geometric meaning. If every maximal ideal is of the shape $\mk{M}_x$, then this condition is equivalent to $\underbrace{f(x)}_{=f+\mk{M}_x} \neq 0$. 

So the Zariski topology is generated by \underline{non-vanishing loci}. 

Why (maximal) spectrum of a Ring? Let $A$ be a \underline{normal} (meaning commutes with its adjoint) matrix/operator. Look at the commutative ring $R$ in $\End_{cts}$ generated by $A,A^\dagger$, take the closure $\bar{R}$. Then $\Specmax \bar{R} = \operatorname{Spec}(A)$, where the right hand side is the functional analysis spectrum of $A$. 

\section*{Lecture 2, 5/9/25}

Last time: Gelfand-Naymark

We had a ``dictionary" relating compact spaces and their function rings. Given an abstract ring of functions, we can reconstruct a compact space. Points correspond to maximal ideals, with the topology generated by preimages $f^{-1}(U)$, where $f:\Specmax R\to\R$ is the map $\mk{M}\mapsto \barf\in R/\mk{M}\cong\R$.

\paragraph{}

Today: \underline{Nullstellensatz} (Hilbert zero theorem)

Aside on etymology: ``Nullstellen" means ``a zero of a function/polynomial", and "satz" means theorem. 

\underline{Fix:} A field $k$, assumed to be 
\begin{itemize}

\item Algebraically closed

\item (for simplicity) uncountable

\end{itemize}

Given a subset $T$ of a polynomial ring over $k$, $T \subseteq R_n \eqdef k[X_1,\dots,X_n]$, we denote by $Z(T)$ the set of common zeroes in $k^n$:
\[
Z(T)\eqdef \{(x = (x_1,\dots,x_n) \mid f(x) = 0\, \forall f \in T\}
\]

The collection of subsets obtained in this way are called ``algebraic sets" by Hartshorne. In this class, we will call them \underline{affine algebraic varieties}. 

\claim

Denoting by $(T)$ the ideal in $R_n$ generated by $T$, we have $Z\left((T)\right) = Z(T)$

\proof

Think

\qed

\underline{Conversely:} Given any subset $S\subseteq k^n$, we may consider the ideal of polynomials in $k_n$ vanishing on $S$. 
\[
\mathcal{I}(S) = \{f\in R_n \mid f(z) = 0\, \forall z \in S\}  
\]
\[
\begin{tikzcd}
\text{alg subsets} \ar[r, dotted, bend left = 30, "\mathcal{I}"] & \text{ideals} \ar[l, dotted, bend left = 30, "Z"]
\end{tikzcd}
\]

\underline{Careful:} 
\begin{itemize}

\item $\mc{I}\left(Z(I)\right) \supset I$

$Z\left(\mc{I}(S)\right) \supset \bar{S}$ (we call $\bar{S}$ the \underline{Zariski closure}, which just means the closure in the Zariski topology)

\end{itemize}

\defn

The \underline{Zariski topology} is defined to be the topology on $k^n$ with closed subsets being the algebraic subsets. 

Reminder: we assume a field $k$ is algebraically complete and uncountable.

\lem

Let $L/k$ be a field extension with $\dim_k(L) \leq |\N|$. Then $L = k$.

\proof

Assume by contradiction that there exists $x \in L\setminus k$. Consider the \underline{uncountable} family given by 
\[
\{\frac{1}{(x-\lambda)} \mid lambda \in k\}
\]
But $\dim_k L \leq |\N|$, so there is a $k$-linear relation. That is, there exists $\lambda_1,\dots,\lambda_r \in k, \mu_1\dots,\mu_r\in k$ so that
\[
\sum_{i=1}^r\frac{\mu_i}{x -\lambda_i} = 0
\]
Clearing the denominators: 
\[
\sum_{i=1}^r \mu_i\prod_{s\neq r}(x-\lambda_s) =0
\]
This is $P(t)$ for some $P$ in $k[t]$. But $k$ is algebraically closed, so $t$ is in $k$, contradiction. 

\qed

\cor[Weak Nullstellensatz]

Let $T \subset R_n$ such that $Z(T) = \varnothing$. Then $(T) = (1) = R_n$.

\proof

Assume by contradiction that $(T) := I \neq R_n$. By Zorn's lemma, there exists a maximal ideal $\mk{M} \supset I$. We look at the chain of quotient maps
\[
R_n \to R_n/I \to \underbrace{R_n/\mk{M}}_{\text{field}} = k
\]

The composition sends $X_i \to x_i \in k$. So $\{R_n \to k\} \supset \{R_n/I \to k\}$. But the former is $k^n$, and the latter is $Z(I)$, which is nonzero, contradicting that $\mk{m}$ is maximal. 

\qed

\underline{Now:} \underline{Rabinowitsch trick}

\lem

Let $T = \{f_1, \dots, f_r\} \subset R_n$ and $f \in \mc{I}\left(Z(T)\right)$, i.e. if $f_i(x) = 0$ for all $i = 1,\dots,r$, then $f(x) = 0$. 

Then there is an $N \in \N$ such that $f^N \in (T)$. 

\proof

Add an auxiliary variable $t$, work with the ring $R_n[t] \equiv R_{n+1}$.

By assumption, $\{(1-tf), f_1, \dots, f_r\} = T'$ doesn't have a common zero, so by weak Nullstellensatz, $(T') = (1)$, so there exists $g_0,\dots,g_r \in R_n[t]$ so that $g_0(1-tf) + g_1f_1 + \dots + g_rf_r = 1$. 

Substitute $t = \frac{1}{f}$, and $g_1f_1 + \dots + g_rf_r = 1$ in a ring of rational functions: $R_n[\frac{1}{f}]$. 

Clearing denominators by multiplying by a sufficiently high power of $f$, we get another expression
\[
\tilde{g_1}f_1 + \dots + \tilde{g_r}f_r = f^N \in R_n
\]
So $f^N \in (T)$. 

\qed

\defn

Let $I \subset R$ be an ideal. We denote by $\sqrt{I} \subset R$ the \underline{radical} of $I$, the set of all $x \in R$ so that $x^n \in I$ for some $n$. 

\thm[Nullstellensatz]

For an ideal $I \subset R$, we have $\mc{I}\left(Z(I)\right) = \sqrt{I}$

\proof

Combine the lemma with the fact that $R_n$ is a Noetherian ring (i.e. ideals are finitely generated). 

\qed

\cor

There is a 1-1 correspondence between affine algebraic $k$-varieities (up to isomorphism) and finitely generated reduced $k$-algebras (up to isomorphism)

\proof

$Z(\sqrt{I})$ corresponds to $R_n / \sqrt{I}$. An isomorphism between varieities is a pair of polynomial maps that map the varieties onto each other and are mutual inverses. 

\qed

There is a stronger version, which gives an equivalence of categories. $\operatorname{AffVar}_k$ is the category whose objects are affine $k$-varieties, and whose morphisms are polynomial maps between ambient spaces preserving the varieties. The category $(\operatorname{Alg}^{red}_k)^\text{op}$ is the opposite category of reduced finitely generated $k$-algebras. The above furnishes an equivalence of these categories. 


\section*{Lecture 3, 8/9/25}

\underline{Today:} Sheaves via \'Etal\'e Spaces

\paragraph{Most textbooks:} 
\begin{itemize}

\item Define presheaves first on a fixed space

\item Then define gluing condition for sections of presheaves

\item Sheaves are defined as presheaves satisfying the gluing condition

\end{itemize}

\'etaler is the French word for ``to spread out." 

Later on, we will encounter the word \'etale, which will appear in the notion of \'etale morphisms of schemes and \'etale cohomology. 

Warning: don't drop the \underline{accent aigue}

\defn

Let $X$ be a topological space. A continuous map $\pi:\mc{S}\to X$ is called a \underline{local homeomorphism} if the following are satisfied:

\begin{itemize}

\item $\pi$ is an open map

\item For every $x \in \mc{S}$, there is an open neighborhood $U\ni x$ such that $\pi|_U:U\to \pi(U)$ is a homeomorphism.

\end{itemize}

In this case, we will say that $\mc{S}$ is \underline{\'etal\'e} above $X$, or call it an \underline{\'etal\'e space}, or simply a sheaf on $X$.

\exm\,

\begin{enumerate}

\item $\varnothing \into X$

\item $\Id_X:X\to X$

\item Let $I$ be a set with the discrete topology. Then $\operatorname{pr}:X\times I \to X$

\item Any covering space, e.g. the M\"obius covering $\mbb{S}^1 \to \mbb{S}^1$ sending $z$ to $z^2$, viewing $\mbb{S}^1$ as a subset of $\C$. 

\item The inclusion $\iota:U\into X$ for any open subset $U$. 

\item For $x \in X$, build a new space by doubling $x$:
\[
X \coprod_{X\setminus\{x\}} X = (X\coprod X)/\sim
\]
There's a natural map $\bigtriangledown$ to $X$, the co-diagonal map. 

\item Let $I \neq \varnothing$ be a set.  
\[
\bigtriangledown: S_{I,x} \eqdef \underbrace{X \coprod_{X\setminus\{x\}} \cdots \coprod_{X\setminus \{x\}} X}_{I\text{ times }}\to X
\]

\end{enumerate}

\underline{Non-example:} 

Take a non-open subset $M \subset X$. Then the inclusion $\iota:M\into X$ is not a local homeomorphism. 

\defn

Let $U\subseteq X$ be open, $\mc{S}$ an \'etal\'e space above $X$. Then \underline{a section on $U$} is a continuous map $s:U\to\mc{S}$ such that $\pi\circ s = \Id_U$. That is, the diagram commutes:
\[
\begin{tikzcd}
& \mc{S} \ar[d, "\pi"] \\
U \ar[ur,"s"] \ar[r,hook,"\iota"] & X
\end{tikzcd}
\]

The set of all sections on $U$ will be denoted by $\mc{S}(U)$ or $\Gamma(U,\mc{S})$. 

If $U = X$, then $s$ is called a \underline{global section}, and we use the notation $\Gamma(\mc{S})$ or $\Gamma(X)$.

Let's revisit the examples above:
\begin{enumerate}

\item 
\[
\begin{tikzcd}
& \varnothing \ar[d, hook] \\
U \ar[ur,dotted,"?"] \ar[r,hook] & X
\end{tikzcd}
\]
If $U$ is nonempty, $\mc{S}(U)$ will be empty, and it will be a singleton if $U$ is empty (namely, $\Id_{\varnothing}:\varnothing\to\varnothing$)

\item 
\[
\begin{tikzcd}
& X \ar[d,"\Id"] \\
U \ar[ur,"\iota",hook] \ar[r,hook] & X
\end{tikzcd}
\]
In this case, $\mc{S}(U) = \{\iota\}$, the inclusion.

\item 
\[
\begin{tikzcd}
& X \times I\ar[d,"pr"] \\
U \ar[ur,"?"] \ar[r,hook] & X
\end{tikzcd}
\]
In this case, $\mc{S}(U) = I$ if $U$ is connected. Otherwise, it is the set of continuous maps from $U$ to $I$, where $I$ carries the discrete topology. We can also think of this as the set of ways to express $U$ as a disjoint union of open subsets indexed by $I$.

\item 
\[
\begin{tikzcd}
& \mc{S}^1\ar[d,"z^2"] \\
U \ar[ur,dotted] \ar[r,hook] & \mc{S}^1
\end{tikzcd}
\]
$\mc{S}(U) = \begin{cases} \varnothing & U = \mc{S}^1 \\ \{*\} & U = \varnothing \\ ? & U \text{ general} \end{cases}$

For $U$ general, $\mc{S}(U)  = \{f:U\to\C \mid \forall z \in U, f(z)^2 = z\}$

\item 
\[
\begin{tikzcd}
& U \ar[d, hook, "\iota_U"] \\
V \ar[r,hook,"\iota_V"] \ar[ur, dotted] & X
\end{tikzcd}
\]
$\mc{S}(V) = \{*\}$ if $V \subset U$, $\varnothing$ otherwise. 

\item 
\[
\begin{tikzcd}
& S = X\coprod_{X\setminus\{x\}} X \ar[d, "\bigtriangledown"] \\
U \ar[ur,dotted,"s"] \ar[r,hook] & X 
\end{tikzcd}
\]
$\mc{S}(U) = \{*\}$ if $U\not\ni x$, otherwise $\{1,2\}$, depending on the choice of which of the two copies of the point $x$. 

\item 
\[
\begin{tikzcd}
& S_{I,x} \ar[d, "\bigtriangledown"] \\
U \ar[ur,dotted,"s"] \ar[r,hook] & X 
\end{tikzcd}
\]
Again, $\mc{S}(U) = \{*\}$ if $U \not\ni x$, and $I$ if $U \ni x$. 

We call this example the ``\underline{skyscraper sheaf at $x$}"

\end{enumerate}

There are many other examples, some even more interesting, which can be described using this theory.

\subsection*{\underline{Holomorphic functions as continuous sections}}

Let $X = \C$ with the standard topology. 

\claim

There exists a space $\ms{H}$ with a local homeomorphism $\pi:X\to\C$ such that continuous sections correspond to holomorphic functions on $\C$, i.e. 
\[
\ms{H}(U) \cong \{f:U\to \C \mid f\text{ holomorphic }\}
\]
compatible with restrictions to smaller open subsets. 

\proof

As a set, 
\[
\ms{H} = \coprod_{z_0\in\C}\{\sum_{n\in\N}c_n(z - z_0)^n \mid \exists r > 0 \text{ the series converges absolutely in a radius $r$ around $z_0$ }\}
\]
The map from $\ms{H}\to\C$ is given by sending a power series which converges in a radius around $z_0$ to $z_0$. 

To get the topology, we choose the strongest topology on $\ms{H}$ such that for every open subset $U$, and every holomorphic function $f:U\to\C$, the induced map $Xf:U\to \ms{H}$ given by $z_0 \mapsto $Taylor$(f, z = z_0)$ is continuous. 

Exercise: Check that $\ms{H}(U) = \{f:U\to \C$ holomorphic $\}$ in a natural way. 

\underline{Remark:} This looks like a generalization of a \underline{phase space} of $\C$ with a real topology:
\[
\R^n \to \R^{2n}, x(t) \mapsto ((x(t), \dot{x}(t))
\]

For this week, read Hartshorne section 2.1 (sheaves)











\end{document}
