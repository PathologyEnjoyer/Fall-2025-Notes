
\documentclass[x11names,reqno,14pt]{extarticle}
\input{preamble}
\usepackage[document]{ragged2e}
\usepackage{epsfig}
\usepackage{dynkin-diagrams}

\pagestyle{fancy}{
	\fancyhead[L]{Fall 2024}
	\fancyhead[C]{MAT1344F}
	\fancyhead[R]{John White}
  
  \fancyfoot[R]{\footnotesize Page \thepage \ of \pageref{LastPage}}
	\fancyfoot[C]{}
	}
\fancypagestyle{firststyle}{
     \fancyhead[L]{}
     \fancyhead[R]{}
     \fancyhead[C]{}
     \renewcommand{\headrulewidth}{0pt}
	\fancyfoot[R]{\footnotesize Page \thepage \ of \pageref{LastPage}}
}
\newcommand{\pmat}[4]{\begin{pmatrix} #1 & #2 \\ #3 & #4 \end{pmatrix}}
\newcommand{\A}{\mathbb{A}}
\newcommand{\B}{\mathbb{B}}
\newcommand{\fin}{``\in"}
\newcommand{\mk}[1]{\mathfrak{#1}}
\newcommand{\g}{\mk{g}}
\newcommand{\h}{\mk{h}}
\newcommand{\J}{\mc{J}}
\newcommand{\tphi}{\tilde{\phi}}
\newcommand{\pois}[2]{\{#1,#2\}}
\newcommand{\fibrate}[3]{\begin{tikzcd} #1 \ar[d, "#2"] \\ #3 \end{tikzcd}}
\renewcommand{\t}{\mk{t}}
\newcommand{\p}{\mk{p}}
\renewcommand{\O}{\mc{O}}
\DeclareMathOperator{\aff}{aff}
\DeclareMathOperator{\Spec}{Spec}
\DeclareMathOperator{\Perm}{Perm}
\DeclareMathOperator{\pdim}{pdim}
\DeclareMathOperator{\gldim}{gldim}
\DeclareMathOperator{\lgldim}{lgldim}
\DeclareMathOperator{\rgldim}{rgldim}
\DeclareMathOperator{\idim}{idim}
\DeclareMathOperator{\SU}{SU}
\DeclareMathOperator{\SO}{SO}
\DeclareMathOperator{\Ad}{Ad}
\DeclareMathOperator{\ad}{ad}
\DeclareMathOperator{\gr}{gr}
\DeclareMathOperator{\Sig}{Sig}
\DeclareMathOperator{\Sh}{Sh}
\DeclareMathOperator{\Psh}{Psh}
\DeclareMathOperator{\LRS}{LocRingdSpc}
\DeclareMathOperator{\Fun}{Fun}
\newcommand{\Rmod}{R-\text{mod}}
\newcommand{\RMod}{R-\text{Mod}}
\newcommand{\onto}{\twoheadrightarrow}
\newcommand{\into}{\hookrightarrow}
\newcommand{\barf}{\bar{f}}
\newcommand{\dd}[2]{\frac{d#1}{d#2}}
\newcommand{\pp}[2]{\frac{\partial #1}{\partial #2}}
\newcommand{\gl}{\mk{g}\mk{l}}
\newcommand{\spew}{\Sp(E,\omega)}
\newcommand{\jew}{\mc{J}(E,\omega)}
\newcommand{\Specmax}{\operatorname{Spec}_{\operatorname{max}}}
\renewcommand{\P}{\mathbb{P}}
\renewcommand{\E}{\mathbb{E}}
\DeclareMathOperator{\Ext}{Ext}
\DeclareMathOperator{\Rank}{Rank}
\DeclareMathOperator{\Sp}{Sp}
\DeclareMathOperator{\ann}{ann}
\DeclareMathOperator{\Lag}{Lag}
\DeclareMathOperator{\Riem}{Riem}
\DeclareMathOperator{\Span}{span}
\newcommand{\exactlon}[5]{
		\begin{tikzcd}
			0\ar[r]&#1\ar[r,"#2"]& #3 \ar[r,"#4"]& #5 \ar[r]&0
		\end{tikzcd}
}

\title{MAT 1190}
\author{John White}
\date{Fall 2025}


\begin{document}

\section*{Lecture 1, 3/9/25 (Happy birthday to me)}

Oh dear, we're starting with chapter 2 of Hartshorne...

Read chapter 1.1 of Hartshorne before friday

Test your understanding of the important bits against Exercise 1.4(Zariski vs product topology)

Following theorem is perhaps unconventional for an ag class.

We use the 	``Bourbaki conventions:"

\defn

A topological space $X$ is said to be \underline{quasicompact} if for every open cover $X = \bigcup_{i\in I}U_i$, there existss a finite subcover $I' \subset I$ such that $X = \bigcup_{i\in I'}U_i$. 

This is usally called ``compact".

\defn

A topological space is said to be \underline{compact} if it is quasicompact and Hausdorff.

\paragraph{Recall:} $X$ is called \underline{Hausdorff} if for all pairs $(x, y)$ of \underline{distinct} points there exist neighborhoods $U_x, U_y$ of $x, y$, such that $U_x \cap U_y = \varnothing$. 

In French, one uses the term ``separated space."

These terms will reappear in algebraic geometry when studying separated schemes.

\paragraph{}

This property is equivalent to the following: For $(x, y) \in X\times X \setminus \bigtriangleup$ (the diagonal elements $\{(x,x) \mid x \in X\}$), there are neighborhoods $U \ni x, V \ni y$, with $U\times V \cap \bigtriangleup = \varnothing$. 

Hence $U\times V$ lies entirely within the complement of the diagonal. So $(x, y)$ is in the interior of $X\times X \setminus \bigtriangleup$. 

Using the definition of the product topology, one can show that $X$ is a Hausdorff space if and only if $\bigtriangleup$ is closed in the product topology. This is the formulation which will be meaningful when we transport to algebraic geometry. 

\thm[Gelfond-Naymark]

Roughly: 

A compact (quasicompact + hausdorff) topological space can be ``recovered" from the ring $C(X) \eqdef C(X,\R)$ of continuous real-valued functions.

\proof

This is a special case of what they proved. Will get into proof later

\qed

In particular, we want it to be true that if $X, Y$ are two compact spaces with abstractly isomorphic rings of functions, i.e. $C(X) \cong C(Y)$, then $X, Y$ should be homeomorphic, $X \cong Y$. 

\paragraph{\underline{From rings to spaces}}

To fix conventions: 

\defn

When we write ``ring", we always mean a commutative, unital ring. So $C(X)$ is indeed always a ring (obviously). 
\paragraph{}

\underline{First step:}

Try to recover the underlying set of points. 

Ideals: given $x \in X$, we obtain a ring homomorphism, called the \underline{evaluation index at $x$}, $e_x:C(X)\to \R$ which takes a continuous real-valued function and evaluates it at $x$: $f\mapsto f(x)$. 

Since $(f+g)(x) = f(x) + g(x)$, and similarly for multiplication, this really is a ring homomorphism. 

\paragraph{Fact:} This map $(e_x)$ is surjective because of constant functions. 

Thus we have the isomorphisms $\R \cong \frac{C(X)}{\ker(e_x)}$. We refer to the denominator as $\mathfrak{M}_x$, the ideal of functions vanishing at $x \in X$. Note that the quotient is a field, so $\mathfrak{M}_x$ is maximal. 

\defn

Let $R$ be a ring. We denote by $\Specmax(R)$ the set of maximal ideals in $R$. 

\prop

Let $X$ be compact. Then there exists a bijection of sets $X \cong \Specmax C(X)$. 

The precise claim may be summarized as follows: 
\begin{itemize}

\item Every maximal ideal $I$ of $C(X)$ is of the form $I = e_x$ for some $x \in X$. 

\item If $x, y$ are points in $X$, and $\mathfrak{M}_x  =\mathfrak{M}_y$, then $x = y$. 

\end{itemize}

\proof
\qed

What about the topology? Let $R$ be an abstract ring with the additional property that for every maximal ideal $\mk{M}\in\Specmax R$, the quotient $R/\mk{M} \cong \R$. Then we can make the following construction: for every element of the ring, we can associate to every element $f \in R$ a function $f:\Specmax R \to \R$ in the following way: $\mk{M}\mapsto \barf \in \R \cong R/\mk{M}$. 

\paragraph{Aside:}

To an algebraist, we think of $\R|(\bar{\Q}\cap\R)$ as a transcendental extension, $\R = (\bar{\Q}\cap \R)(\alpha_0,\alpha_1,\dots)$. So, there are lots of field automorphisms on $\R$, none of which are continuous. 

Aside over. 

\paragraph{Now:} Look at the coarsest topology on $\Specmax R$ such that all functions $\mk{M} \mapsto f + \mk{M} \in \R$ are continuous for each $f \in R$. 

That is, the topology on $\Specmax R$ is generated by preimages $f^{-1}(U)$, where $f:\Specmax R\to\R$ denotes the map associated with $f \in R$. 

Due to the existence of noncontinuous elements of $\Aut(\R)$, it is problematic to work with the standard topology. 

It is in some way ``unnatural" to think of the topology of $\R$ analyticaly, if we want to do algebra. 

\paragraph{Instead:} We use the cofinite topology on $\R$ instead, i.e. the nonempty open subsets are the complements of finite sets. 

\defn

Let $R$ be a ring. Then the \underline{Zariski topology} on $\Specmax R$ is the topology generated by ``standard open subsets," which are defined as subsets of the form 
\[
U_f = \{\mk{M} \in \Specmax R \mid f\not\in \mk{M}\}
\]

It is in a certain way ``algebraically robust". 

\paragraph{Remark:} The condition that $f \not \in \mk{M}$ has a very geometric meaning. If every maximal ideal is of the shape $\mk{M}_x$, then this condition is equivalent to $\underbrace{f(x)}_{=f+\mk{M}_x} \neq 0$. 

So the Zariski topology is generated by \underline{non-vanishing loci}. 

Why (maximal) spectrum of a Ring? Let $A$ be a \underline{normal} (meaning commutes with its adjoint) matrix/operator. Look at the commutative ring $R$ in $\End_{cts}$ generated by $A,A^\dagger$, take the closure $\bar{R}$. Then $\Specmax \bar{R} = \operatorname{Spec}(A)$, where the right hand side is the functional analysis spectrum of $A$. 

\section*{Lecture 2, 5/9/25}

Last time: Gelfand-Naymark

We had a ``dictionary" relating compact spaces and their function rings. Given an abstract ring of functions, we can reconstruct a compact space. Points correspond to maximal ideals, with the topology generated by preimages $f^{-1}(U)$, where $f:\Specmax R\to\R$ is the map $\mk{M}\mapsto \barf\in R/\mk{M}\cong\R$.

\paragraph{}

Today: \underline{Nullstellensatz} (Hilbert zero theorem)

Aside on etymology: ``Nullstellen" means ``a zero of a function/polynomial", and "satz" means theorem. 

\underline{Fix:} A field $k$, assumed to be 
\begin{itemize}

\item Algebraically closed

\item (for simplicity) uncountable

\end{itemize}

Given a subset $T$ of a polynomial ring over $k$, $T \subseteq R_n \eqdef k[X_1,\dots,X_n]$, we denote by $Z(T)$ the set of common zeroes in $k^n$:
\[
Z(T)\eqdef \{(x = (x_1,\dots,x_n) \mid f(x) = 0\, \forall f \in T\}
\]

The collection of subsets obtained in this way are called ``algebraic sets" by Hartshorne. In this class, we will call them \underline{affine algebraic varieties}. 

\claim

Denoting by $(T)$ the ideal in $R_n$ generated by $T$, we have $Z\left((T)\right) = Z(T)$

\proof

Think

\qed

\underline{Conversely:} Given any subset $S\subseteq k^n$, we may consider the ideal of polynomials in $k_n$ vanishing on $S$. 
\[
\mathcal{I}(S) = \{f\in R_n \mid f(z) = 0\, \forall z \in S\}  
\]
\[
\begin{tikzcd}
\text{alg subsets} \ar[r, dotted, bend left = 30, "\mathcal{I}"] & \text{ideals} \ar[l, dotted, bend left = 30, "Z"]
\end{tikzcd}
\]

\underline{Careful:} 
\begin{itemize}

\item $\mc{I}\left(Z(I)\right) \supset I$

$Z\left(\mc{I}(S)\right) \supset \bar{S}$ (we call $\bar{S}$ the \underline{Zariski closure}, which just means the closure in the Zariski topology)

\end{itemize}

\defn

The \underline{Zariski topology} is defined to be the topology on $k^n$ with closed subsets being the algebraic subsets. 

Reminder: we assume a field $k$ is algebraically complete and uncountable.

\lem

Let $L/k$ be a field extension with $\dim_k(L) \leq |\N|$. Then $L = k$.

\proof

Assume by contradiction that there exists $x \in L\setminus k$. Consider the \underline{uncountable} family given by 
\[
\{\frac{1}{(x-\lambda)} \mid lambda \in k\}
\]
But $\dim_k L \leq |\N|$, so there is a $k$-linear relation. That is, there exists $\lambda_1,\dots,\lambda_r \in k, \mu_1\dots,\mu_r\in k$ so that
\[
\sum_{i=1}^r\frac{\mu_i}{x -\lambda_i} = 0
\]
Clearing the denominators: 
\[
\sum_{i=1}^r \mu_i\prod_{s\neq r}(x-\lambda_s) =0
\]
This is $P(t)$ for some $P$ in $k[t]$. But $k$ is algebraically closed, so $t$ is in $k$, contradiction. 

\qed

\cor[Weak Nullstellensatz]

Let $T \subset R_n$ such that $Z(T) = \varnothing$. Then $(T) = (1) = R_n$.

\proof

Assume by contradiction that $(T) := I \neq R_n$. By Zorn's lemma, there exists a maximal ideal $\mk{M} \supset I$. We look at the chain of quotient maps
\[
R_n \to R_n/I \to \underbrace{R_n/\mk{M}}_{\text{field}} = k
\]

The composition sends $X_i \to x_i \in k$. So $\{R_n \to k\} \supset \{R_n/I \to k\}$. But the former is $k^n$, and the latter is $Z(I)$, which is nonzero, contradicting that $\mk{m}$ is maximal. 

\qed

\underline{Now:} \underline{Rabinowitsch trick}

\lem

Let $T = \{f_1, \dots, f_r\} \subset R_n$ and $f \in \mc{I}\left(Z(T)\right)$, i.e. if $f_i(x) = 0$ for all $i = 1,\dots,r$, then $f(x) = 0$. 

Then there is an $N \in \N$ such that $f^N \in (T)$. 

\proof

Add an auxiliary variable $t$, work with the ring $R_n[t] \equiv R_{n+1}$.

By assumption, $\{(1-tf), f_1, \dots, f_r\} = T'$ doesn't have a common zero, so by weak Nullstellensatz, $(T') = (1)$, so there exists $g_0,\dots,g_r \in R_n[t]$ so that $g_0(1-tf) + g_1f_1 + \dots + g_rf_r = 1$. 

Substitute $t = \frac{1}{f}$, and $g_1f_1 + \dots + g_rf_r = 1$ in a ring of rational functions: $R_n[\frac{1}{f}]$. 

Clearing denominators by multiplying by a sufficiently high power of $f$, we get another expression
\[
\tilde{g_1}f_1 + \dots + \tilde{g_r}f_r = f^N \in R_n
\]
So $f^N \in (T)$. 

\qed

\defn

Let $I \subset R$ be an ideal. We denote by $\sqrt{I} \subset R$ the \underline{radical} of $I$, the set of all $x \in R$ so that $x^n \in I$ for some $n$. 

\thm[Nullstellensatz]

For an ideal $I \subset R$, we have $\mc{I}\left(Z(I)\right) = \sqrt{I}$

\proof

Combine the lemma with the fact that $R_n$ is a Noetherian ring (i.e. ideals are finitely generated). 

\qed

\cor

There is a 1-1 correspondence between affine algebraic $k$-varieities (up to isomorphism) and finitely generated reduced $k$-algebras (up to isomorphism)

\proof

$Z(\sqrt{I})$ corresponds to $R_n / \sqrt{I}$. An isomorphism between varieities is a pair of polynomial maps that map the varieties onto each other and are mutual inverses. 

\qed

There is a stronger version, which gives an equivalence of categories. $\operatorname{AffVar}_k$ is the category whose objects are affine $k$-varieties, and whose morphisms are polynomial maps between ambient spaces preserving the varieties. The category $(\operatorname{Alg}^{red}_k)^\text{op}$ is the opposite category of reduced finitely generated $k$-algebras. The above furnishes an equivalence of these categories. 


\section*{Lecture 3, 8/9/25}

\underline{Today:} Sheaves via \'Etal\'e Spaces

\paragraph{Most textbooks:} 
\begin{itemize}

\item Define presheaves first on a fixed space

\item Then define gluing condition for sections of presheaves

\item Sheaves are defined as presheaves satisfying the gluing condition

\end{itemize}

\'etaler is the French word for ``to spread out." 

Later on, we will encounter the word \'etale, which will appear in the notion of \'etale morphisms of schemes and \'etale cohomology. 

Warning: don't drop the \underline{accent aigue}

\defn

Let $X$ be a topological space. A continuous map $\pi:\mc{S}\to X$ is called a \underline{local homeomorphism} if the following are satisfied:

\begin{itemize}

\item $\pi$ is an open map

\item For every $x \in \mc{S}$, there is an open neighborhood $U\ni x$ such that $\pi|_U:U\to \pi(U)$ is a homeomorphism.

\end{itemize}

In this case, we will say that $\mc{S}$ is \underline{\'etal\'e} above $X$, or call it an \underline{\'etal\'e space}, or simply a sheaf on $X$.

\exm\,

\begin{enumerate}

\item $\varnothing \into X$

\item $\Id_X:X\to X$

\item Let $I$ be a set with the discrete topology. Then $\operatorname{pr}:X\times I \to X$

\item Any covering space, e.g. the M\"obius covering $\mbb{S}^1 \to \mbb{S}^1$ sending $z$ to $z^2$, viewing $\mbb{S}^1$ as a subset of $\C$. 

\item The inclusion $\iota:U\into X$ for any open subset $U$. 

\item For $x \in X$, build a new space by doubling $x$:
\[
X \coprod_{X\setminus\{x\}} X = (X\coprod X)/\sim
\]
There's a natural map $\bigtriangledown$ to $X$, the co-diagonal map. 

\item Let $I \neq \varnothing$ be a set.  
\[
\bigtriangledown: S_{I,x} \eqdef \underbrace{X \coprod_{X\setminus\{x\}} \cdots \coprod_{X\setminus \{x\}} X}_{I\text{ times }}\to X
\]

\end{enumerate}

\underline{Non-example:} 

Take a non-open subset $M \subset X$. Then the inclusion $\iota:M\into X$ is not a local homeomorphism. 

\defn

Let $U\subseteq X$ be open, $\mc{S}$ an \'etal\'e space above $X$. Then \underline{a section on $U$} is a continuous map $s:U\to\mc{S}$ such that $\pi\circ s = \Id_U$. That is, the diagram commutes:
\[
\begin{tikzcd}
& \mc{S} \ar[d, "\pi"] \\
U \ar[ur,"s"] \ar[r,hook,"\iota"] & X
\end{tikzcd}
\]

The set of all sections on $U$ will be denoted by $\mc{S}(U)$ or $\Gamma(U,\mc{S})$. 

If $U = X$, then $s$ is called a \underline{global section}, and we use the notation $\Gamma(\mc{S})$ or $\Gamma(X)$.

Let's revisit the examples above:
\begin{enumerate}

\item 
\[
\begin{tikzcd}
& \varnothing \ar[d, hook] \\
U \ar[ur,dotted,"?"] \ar[r,hook] & X
\end{tikzcd}
\]
If $U$ is nonempty, $\mc{S}(U)$ will be empty, and it will be a singleton if $U$ is empty (namely, $\Id_{\varnothing}:\varnothing\to\varnothing$)

\item 
\[
\begin{tikzcd}
& X \ar[d,"\Id"] \\
U \ar[ur,"\iota",hook] \ar[r,hook] & X
\end{tikzcd}
\]
In this case, $\mc{S}(U) = \{\iota\}$, the inclusion.

\item 
\[
\begin{tikzcd}
& X \times I\ar[d,"pr"] \\
U \ar[ur,"?"] \ar[r,hook] & X
\end{tikzcd}
\]
In this case, $\mc{S}(U) = I$ if $U$ is connected. Otherwise, it is the set of continuous maps from $U$ to $I$, where $I$ carries the discrete topology. We can also think of this as the set of ways to express $U$ as a disjoint union of open subsets indexed by $I$.

\item 
\[
\begin{tikzcd}
& \mc{S}^1\ar[d,"z^2"] \\
U \ar[ur,dotted] \ar[r,hook] & \mc{S}^1
\end{tikzcd}
\]
$\mc{S}(U) = \begin{cases} \varnothing & U = \mc{S}^1 \\ \{*\} & U = \varnothing \\ ? & U \text{ general} \end{cases}$

For $U$ general, $\mc{S}(U)  = \{f:U\to\C \mid \forall z \in U, f(z)^2 = z\}$

\item 
\[
\begin{tikzcd}
& U \ar[d, hook, "\iota_U"] \\
V \ar[r,hook,"\iota_V"] \ar[ur, dotted] & X
\end{tikzcd}
\]
$\mc{S}(V) = \{*\}$ if $V \subset U$, $\varnothing$ otherwise. 

\item 
\[
\begin{tikzcd}
& S = X\coprod_{X\setminus\{x\}} X \ar[d, "\bigtriangledown"] \\
U \ar[ur,dotted,"s"] \ar[r,hook] & X 
\end{tikzcd}
\]
$\mc{S}(U) = \{*\}$ if $U\not\ni x$, otherwise $\{1,2\}$, depending on the choice of which of the two copies of the point $x$. 

\item 
\[
\begin{tikzcd}
& S_{I,x} \ar[d, "\bigtriangledown"] \\
U \ar[ur,dotted,"s"] \ar[r,hook] & X 
\end{tikzcd}
\]
Again, $\mc{S}(U) = \{*\}$ if $U \not\ni x$, and $I$ if $U \ni x$. 

We call this example the ``\underline{skyscraper sheaf at $x$}"

\end{enumerate}

There are many other examples, some even more interesting, which can be described using this theory.

\subsection*{\underline{Holomorphic functions as continuous sections}}

Let $X = \C$ with the standard topology. 

\claim

There exists a space $\ms{H}$ with a local homeomorphism $\pi:X\to\C$ such that continuous sections correspond to holomorphic functions on $\C$, i.e. 
\[
\ms{H}(U) \cong \{f:U\to \C \mid f\text{ holomorphic }\}
\]
compatible with restrictions to smaller open subsets. 

\proof

As a set, 
\[
\ms{H} = \coprod_{z_0\in\C}\{\sum_{n\in\N}c_n(z - z_0)^n \mid \exists r > 0 \text{ the series converges absolutely in a radius $r$ around $z_0$ }\}
\]
The map from $\ms{H}\to\C$ is given by sending a power series which converges in a radius around $z_0$ to $z_0$. 

To get the topology, we choose the strongest topology on $\ms{H}$ such that for every open subset $U$, and every holomorphic function $f:U\to\C$, the induced map $Xf:U\to \ms{H}$ given by $z_0 \mapsto $Taylor$(f, z = z_0)$ is continuous. 

Exercise: Check that $\ms{H}(U) = \{f:U\to \C$ holomorphic $\}$ in a natural way. 

\underline{Remark:} This looks like a generalization of a \underline{phase space} of $\C$ with a real topology:
\[
\R^n \to \R^{2n}, x(t) \mapsto ((x(t), \dot{x}(t))
\]

For this week, read Hartshorne section 2.1 (sheaves)

\section*{Lecture 4, 10/9/25}

\underline{Today:} Stalks

On Monday, we did sheaf theory via \'etal\'e spaces. 

We define a sheaf as a continuous map $\pi:\mc{S}\to X$ which is a local homeomorphism. In this case we say $\mc{S}$ is an \'etal\'e space, or simply a sheaf on $X$.

\defn

Let $X$ be a space, $\pi:\mc{S}\to X$ an \'etal\'e space over $X$. For every $x \in X$ we denote the preimae $\pi^{-1}(x)$ by $\mc{S}_x$ and call it the \underline{stalk of $\mc{S}$ at $x$}.

Do NOT call it a fiber! (We will use this terminology for something different later)

\exm
\,
\begin{enumerate}

\item When $\mc{S} = \varnothing \into X$, for all $x$, $\mc{S}_x = \varnothing$. 

\item When $\mc{S} = X$, $\pi= \Id:X\to X$, $\mc{S}_x = \{x\}$, a singleton. 

\item When $\mc{S} = X\times I$, for a discrete space $I$, $pr:X\times I \to X$ the projection map, we have $\mc{S}_x \cong I$.

\item When $\mc{S} = S_{I,x}$, the skyscraper sheaf at $x$, $\bigtriangledown: S_{I,x} \to X$, we have $\mc{S}_y = \{*\}$ a singleton if $y \neq x$, and $\mc{S}_x = I$

\item Consider the space $\ms{H}\to \C = X$ defined last time, the sheaf of holomorphic functions. Then 
\[
\ms{H}_x = \{\sum_{k=0}^\oo c_k(z-z_0)^k \mid \exists \varepsilon>0\text{ the sum converges in a ball of radius }\varepsilon\text{ around }z_0\}
\]

\end{enumerate}

\lem
\,
\begin{itemize}

\item[Existence:] Let $\pi:\mc{S}\to X$ be an \'etal\'e space over $X$, $x \in X$, and let $y \in \mc{S}_x$ be an element of the stalk. Then there exists an open neighborhood $U \ni x$ and section $s \in \mc{S}(U)$, $s:U\to\mc{S}$, such that $s(x) = y$.

\item[Uniqueness:] Further, given two pairs $(U_1, s_1),(U_2,s_2)$ with this property, then there exists $V \subset U_1 \cap U_2$ such that $s_1|_V = s_2|_V$.

\end{itemize}

\proof

Left as an exercise. Hint: use that $\pi$ is a local homeomorphism. 

\underline{Categorical reformulation:}

Consider the collection of all neighborhoods of $x$, ordered by inclusion, and take 
\[
ev_x: \operatorname{colim}_{U\ni x\text{ open}} \mc{S}(U) \to \mc{S}_x
\]
Then this is a bijection.

In the case of sets, we can describe the right hand side as equivalence classes of pairs $\{(U,s)\}$, $U\ni x$ open, $s\in \mc{S}(U)$, where $(U,s) \sim (V,t)$ if there exists an open $W \subseteq U\cap V$ such that $s|_W = t|_W$.

This colimit corresponds to the set of germs of sections near $x$.

\lem
\,
\begin{enumerate}

\item Let $f:X\to Y, g:Y\to Z$ be composable continuous maps. Denote by $h$ their composition, $h = g\circ f$. Then if $f, g$ are local homeomorphisms, then $h$ is a local homeomorphism. 

\item If $g$ and $h$ are local homeomorphisms, then $f$ is a local homeomorphism.

\end{enumerate}

\proof

Omitted

\qed

\defn[Category of sheaves]

Let $X$ be a topological space. Then the \underline{Category of sheaves on $X$, $Sh(X)$}, is defined to have as its objects the \'etal\'e spaces over $X$, and morphisms defined to be those $\varphi:\mc{S}_1\to\mc{S}_2$ making the diagram commute:
\[
\begin{tikzcd}
\mc{S}_1 \ar[dr,"\pi_1"] \ar[rr, "\varphi"] && \mc{S}_2 \ar[dl, "\pi_2"']  \\
& X & \\
\end{tikzcd}
\]

\lem[Isomorphism criterion]

Let $\phi:\mc{S}_1\to\mc{S}_2$ be a morphism in $Sh(X)$. Then $\varphi$ is an isomorphism if and only if $(\mc{S}_1)_x\to(\mc{S}_2)_x$ is bijective for all $x \in X$. 

\proof

Suppose $\varphi:\mc{S}_1\to\mc{S}_2$ is a bijection of sets. Bijective continuous open maps are homeomorphisms, thus there is an inverse in $Sh(X)$. Other direction is clear. 

\qed

\lem[Injectivity criterion]

The above holds replacing bijection with injection.

\proof

\qed

We can restate in terms of sections.

\lem

Let $\varphi:\mc{S}_1\to\mc{S}_2$ be a morphism in $Sh(X)$ such that for every $U\subseteq X$ open, the induced map $\mc{S}_1(U)\to\mc{S}_2(U)$ is a bijection. Then $\varphi$ is an isomorphism.

\proof

Apply the isomorphism criterion, 
\[
\begin{tikzcd}
(\mc{S}_1)_x \ar[r, "\cong"] \ar[d, "\varphi","\cong"'] & colim_{U\ni x}\mc{S}_1(U) \ar[d,"\cong"] \\
(\mc{S}_2)_x \ar[r, "\cong"] & colim_{U\ni x}\mc{S}_2(U)
\end{tikzcd}
\]
So the induced maps on every stalk is an iso, so $\varphi$ is an isomorphism.

This also works with injection. 

\qed
\,

We expect the same to hold for surjections. That is, we would hope that if $\varphi:\mc{S}_1\to\mc{S}_2$ is surjective, then for all $U \subset X$, the induced map $\mc{S}_1(U)\to\mc{S}_2(U)$ is surjective. This is \underline{false!}
\,
\,
\,
\dbend

\underline{Counterexample:}

Let $X = \mbb{S}^1$. We have the sheaf $\Id_X:X\to X$. It has the M\"obius automorphism $z \mapsto z^2$, which is also a sheaf over $X$:
\[
\begin{tikzcd}
\mc{S} \ar[r] \ar[dr] & X \ar[d,"\Id_X"] \\
& X
\end{tikzcd}
\]
If both unlabeled maps are $z\mapsto z^2$, then the upper map is a surjective map of \'etal\'e spaces, but $\mc{S}(X) = \varnothing$ does not surject onto $X(X) = \{*\}$. 

\lem[Local lifts exist]

Given a surjection $\mc{S}_1\to\mc{S}_2$ in $Sh(X)$, and open $U \subseteq X$, and a section $s \in \mc{S}_2(U)$, there exists an open cover $U = \bigcup_{i\in I}U_i$ and sections $t_i \in \mc{S}_1(U_i)$ such that $\varphi(t_i) = s|_{u_i}$ for all $i$. 

\proof

Let $\varphi$ be a surjective map of \'etal\'e spaces. For all $x \in X$, $\varphi:(\mc{S}_1)_x \to (\mc{S}_2)_x$ is surjective. We take the element $[(s,U)] \in (\mc{S}_2)_x$, which has a preimage $[(t,V)]$. We can repeat this for every $x \in U_i$ to obtain the collection of pairs $(U_i,t_i)$.

\underline{Abstract perspective:}

Consider the commutative triangle 
\[
\begin{tikzcd}
\mc{S}_n \ar[dr] \ar[r,"\varphi"] & \mc{S}_2 \ar[d] \\
& X \ar[u, "s"', bend right = 30] 
\end{tikzcd}
\]

With $s$ a global section. Then $s \in \mc{S}_2$ can be lifted to $t \in \mc{S}_1(X)$ if and only if $s^{-1}\mc{S}_1$ has a global section

\section*{Lecture 4, 12/9/25}

\underline{Today:} Fiber products (of spaces), preimage and pushforward sheaves, presheaves. 

Recall: a sheaf on $X$ is the same thing as an \'etal\'e space over $X$, a topological space $\mc{S}$ with a local homeomorphism $\pi:\mc{S}\to X$, and a morphism of sheaves is a map $\varphi:\mc{S}_1\to\mc{S}_2$ making the diagram commute:
\[
\begin{tikzcd}
\mc{S}_1 \ar[rr,"\varphi"]\ar[dr] && \mc{S}_2 \ar[dl] \\
& X & 
\end{tikzcd}
\]
Note that $\varphi$ must also be a local homeomorphism.
We define the stalk at a point $x$, $\mc{S}_x$, as simply the preimage $\pi^{-1}(x) \subseteq \mc{S}$. 

\subsection*{\underline{Fiber products}}

Given a diagram of continuous maps of topological spaces
\[
\begin{tikzcd}
X\times_Y Z \ar[r] \ar[d] & Z \ar[d,"g"] \\
X \ar[r,"f"] & Y
\end{tikzcd}
\]
The top left $X\times_YZ \eqdef \{(x,z) \in X\times Z \mid f(x) = g(z) \}$ endowed with the subspace topology in $X\times Z$. 

\underline{Universal property:}
\[
\begin{tikzcd}
T \ar[drr, bend left = 30, "\beta"] \ar[dr,dotted,"\exists!"]\ar[ddr, bend right = 30, "\alpha"] & & \\
& X\times_Y Z \ar[r] \ar[d] & Z \ar[d,"g"] \\
& X \ar[r,"f"] & Y
\end{tikzcd}
\]
Given an $\alpha,\beta$ as above making the diagram commute, there is a unique map from $T$ to $X\times_YZ$ making the diagram commute. 
\exm
\,
\begin{enumerate}

\item The usual product:
\[
\begin{tikzcd}
X\times Z \ar[r] \ar[d] & Z \ar[d,"g"] \\
X \ar[r,"f"]& \{*\} 
\end{tikzcd}
\]

\item The fiber above a point $y$:
\[
\begin{tikzcd}
f^{-1}(y) \ar[r]\ar[d] & \{*\}\ar[d, "y"] \\
X \ar[r, "f"] & Y 
\end{tikzcd}
\]
\end{enumerate}

\underline{Preimage-sheaf:}

Given a continuous $f:Y\to X$, we have a functor $f^{-1}: \Sh(X)\to \Sh(Y)$, 
\[
(\pi:\mc{S}\to X) \mapsto (\pi':\mc{S}\times_X Y \to Y)
\]
\[
\begin{tikzcd}
\mc{S}\times_X Y \ar[r] \ar[d, "\pi'"] & \mc{S} \ar[d,"\pi"] \\
Y \ar[r,"f"] & X 
\end{tikzcd}
\]

\lem

$\pi'$ in the above is indeed a sheaf

\proof

Chase definitions

\qed

\underline{Remark:} $f^{-1}$ preserves stalks: that is, for all $y \in Y$, $(f^{-1}\mc{S})_y\cong \mc{S}_{f(y)}$

We also have a functor going the other direction, $f_*:\Sh(Y)\to\Sh(X)$. So given $f:Y\to X$, $\mc{S} \mapsto f_*\mc{S}$. This is called the \underline{pushforward}. 

Given a sheaf $\pi:\mc{S}\to Y$, we want a sheaf $\tilde{\mc{S}}$ and $\tilde{\pi}:\tilde{\mc{S}}\to X$, as well as a function $\tilde{f}:\mc{S}\to\tilde{\mc{S}}$ so that the diagram commutes:
\[
\begin{tikzcd}
\mc{S}\ar[r, "\tilde{f}"] \ar[d, "\pi"] & \tilde{\mc{S}} \ar[d,"\tilde{\pi}"] \\
Y \ar[r, "f"] & X
\end{tikzcd}
\]
It is not immediately clear at all how to construct such a thing. Note that $\tilde{f}$ need not be a local homeomorphism here. 

\underline{Presheaves}

Let $X$ be a space. Denote by $\operatorname{Open}(X)$ the category of open subsets of $X$, with inclusions as morphisms. That is $\Hom_{U,V} = \{*\}$ if $V\subseteq U$, and $\Hom_{U,V} = \varnothing$ otherwise. 

\defn

A \underline{presheaf} on $X$ is defined to be a functor from 
\[
\mc{F}:\operatorname{Open}(X)^{op} \to \Set
\]

\underline{Concretely:}

For all $U \subset X$, specify a set $\mc{F}(U)$, such that for all $V \subset U$ inclusions, there is a restriction map $r^U_V:\mc{F}(U)\to\mc{F}(V)$ such that the following properties hold: 
\begin{itemize}

\item $r^U_U = \Id_{\mc{F}(U)}$

\item For $W \subseteq V \subseteq U$, we want 
\[
r^V_W \circ r^U_V = r^U_W
\]
So restricting from $U$ to $V$, and then from $V$ to $W$, is the same as just restricting straight from $U$ to $W$.
\end{itemize}

There is a category of presheaves on $X$, which we denote by $\Psh(X)$, which we can define as 
\[
\Psh(X) \eqdef \Fun(\operatorname{Open}(X)^{op},\Set)
\]

There is a functor $I: \Sh(X)$ to $\Psh(X)$ sending $\pi:\mc{S}\to X$ to the map sending an open $U \subseteq X$ to its set of sections, $\mc{S}(U)$.

One can verify this is indeed a presheaf.

\paragraph{Surprisingly:} Of more interest to us is the existence of a functor $^+:\Psh(X)\to\Sh(X)$, called the presheaf's \underline{associated sheaf}, or its \underline{sheafification}, which interacts nicely with $I$, in the sense that $^+\circ I \simeq \Id_{\Sh(X)}$, and $^+\circ^+ \simeq ^+$

It takes a presheaf $\mc{F}$ and sends it to a sheaf $\mc{F}^+$. This implies that $I$ is an emebedding of categories. So passing from the \'etal\'e space to the presheaf of sections loses no information. 

\subsection*{\underline{Construction of the sheafification:}}

Let $\mc{F} \in \Psh(X)$. Let's first construct the set of points of an \'etal\'e space on $X$. We define the stalk of a presheaf as follows. 

\paragraph{} For any $x \in X$, we define 
\[
\mc{F}_x\eqdef \operatorname{colim}_{U\ni x\text{ open }}\mc{F}(U)
\]
Note that for this to make sense we do need the functoriality of $\mc{F}$. We can define them as germs of sections in exactly the same way, where a ``section" over $U$ is just an element of $\mc{F}(U)$. 

This also gives a clear morphism $\pi:\underbrace{\coprod_{x\in X}}_{\eqdef \mc{S}}\mc{F}_x\to X$.

We now topologize $\mc{S}$. For a topoloical space $T$, every map $f:\mc{S}\to T$ is continuous if and only if for all $U \subseteq X$, for all $s \in \mc{F}(U)$, the composition $f\circ s:U\to T$ is continuous:
\[
\begin{tikzcd}
T& \ar[l]\mc{S} \ar[d,"\pi"] \\
U\ar[u]\ar[ur, dotted, "s"] \ar[r,hook]& X 
\end{tikzcd}
\]
For any neighborhood $V$ of $x$, and $s \in \mc{F}(V)$, we have a map $s:V\to \mc{S}$ given by $y\mapsto s_y$, the image of $s$ in the colimit definition of the stalk at $y$.
So we topologize $\mc{S}$ in the weakest way so that this is the case. Using this definition, we can check that $\pi$ is continuous. All sections $s \in \mc{F}(U)$ give rise to continuous sections of the \'etal\'e space $\mc{S}$. 

\underline{Remark:} The topology on $\mc{S}$ is generated by sets of the form $s(U)$ for all $s\in \mc{F}(U)$ for all $U$ open. 

One can check that $\pi$ is a \underline{local homeomorphism}. 

\claim

If $\pi:\mc{S}\to X$ is an \'etal\'e space then the sheafification of the presheaf of sections of $\mc{S}$ agrees with $\mc{S}$. 

\proof

Recall the lemma that $\pi^{-1}(x)$ can be described as a colimit. So the map $\mc{S} \to \coprod_{x\in X}\mc{S}_x$ (where the $\mc{S}$ on the left-hand side is the presheaf associated to $\mc{S}$) is a continuous bijection. 

Homeomorphisms are precisely the continuous maps which are bijective and open, and one can check that this map is open by construction of the presheaf associated to $\mc{S}$. 

\qed

\subsection*{\underline{Pushforward}}

Given a continuous map $f:Y\to X$, we have the functor $f_*:\Psh(Y)\to\Psh(X)$, given by 
\[
\mc{F} \mapsto \left((U\subseteq X) \mapsto (\mc{F}(f^{-1}(U))\right)
\]
So $f_*(\mc{F}) = F\circ f^{-1}$, where $f^{-1}$ is the functor sending $\operatorname{Open}(X) \to \operatorname{Open}(Y)$.

\claim

$f^*(\Sh(Y)) \subset \Sh(X)$

\proof

Future assignment.

\qed

\section*{Lecture 5, 15/9/25}

\underline{Last time: presheaves and pushforwards}

\underline{Today:} Sheaves $\subset$ presheaves, locally ringed spaces. 

\underline{Reading assignment for this week:} Hartshorne section 2.2, up to and including example 2.3.3.

\prop

Let $X$ be a space, and let $\mc{F} \in \Psh(X)$ be a presheaf on $X$. 

Then the canonical map $\mc{F}\to\mc{F}^+$ is an isomorphism if and only if for every open subset $U$, and every open cover $U = \bigcup_{i\in I}U_i$, the following is an \underline{equalizer}:
\[
\begin{tikzcd}
\mc{F}(U) \ar[r] & \prod_{i\in I}\mc{F}(U_i) \ar[r, bend left = 20, "r^{U_i}_{U_{ij}}"]\ar[r, bend right = 20, "r^{U_j}_{U_{ij}}"'] & \prod_{(i,j) \in I^2}\mc{F}(U_{ij}) 
\end{tikzcd}
\]
where $U_{ij} = U_i \cap U_j$. 

\proof

In a minute

\qed

\underline{Translation:} Given a collection of sections $s_i \in\mc{F}(U_i)$ such that for all $i,j$ we have $r^{U_i}_{U_{ij}}(s_i) = r^{U_j}_{U_{ij}}(s_j)$, then there is a unique section $s \in \mc{F}(U)$ such that $r^U_{U_{ij}} = s_i$ for all $i$. 

\underline{Remark:}
\[
\begin{tikzcd}
A \ar[r] & B \ar[r, bend left = 10,"f"] \ar[r, bend right = 10,"g"'] & C
\end{tikzcd}
\]

is called an \underline{equalizer} in a category $\mc{C}$ if
\[
\begin{tikzcd}
A \ar[r] \ar[d]& B \ar[d, "f\times g"]\\
C \ar[r,"\bigtriangleup"] & C^2
\end{tikzcd}
\]
is a pullback, i.e. $A \simeq C\times_{C\times C}B$. So the map from $A$ to $B$ is a universal map making $f, g$ equal in the composition. 

Now for the proof of the proposition, which will require a lemma. 

\lem

Let $\mc{F} \in \Psh(X)$ such that $\mc{F}$ the gluing condition. Let $s_{1,2}\in\mc{F}(U)$ be local sections such that for all $x \in U$ we have
\[
(s_1)_x = (s_2)_x \in \mc{F}_x \eqdef \operatorname{colim}_{U\ni x}\mc{F}(U)
\]

Then $s_1 = s_2$. 

\proof

For all $x$ there is a neighborhood $x \in U_x \subset X$ such that $r^U_{U_x}(s_1) = r^U_{U_x}(s_2)$. Then $s_1$ and $s_2$ re glue to the local sections $s_x \eqdef r^U_{U_x}(s_1)$ or $r^U_{U_x}(s_2)$. By uniqueness, $s_1 = s_2$. 

\qed

Now for the proof of the proposition. 

\proof
\,
\underline{Easier direction:}

Sections of \'etal\'e spaces satisfy the gluing condition just because of their nature as functions.

\underline{Harder direction:}

We want to show that if the gluing condition is satisfied, then $\mc{F} \simeq \mc{F}^+$. 

We have a presheaf $\mc{F}$ and \'etal\'e space $\mc{S}\to X$, the presheaf of sections $\mc{F}^+$. 

Take $s \in \mc{F}^+(U)$ an arbitrary section, by definition for all $x \in U$, $s(x) \in \mc{F}_x = \mc{S}_x$. Thus there exists an open neighborhood $U_x\ni x$ and a section $s_x:U_x\to\mc{S}$ such that $s_x(x) = s(x)$. Since $\pi:\mc{S}\to X$ is a local homeomorphism, we may further shrink the neighborhood $U_x$ to ensure that $s_x|_{U_x} = s|_{U_x}$. 

Now we apply the lemma to $U_x \cap U_y$ to obtain $s_x|_{U_{xy}} = s_y|_{U_{xy}}$ for all pairs of points $(x,y)$. Because $\mc{F}$ is assumed to satisfy the gluing condition, this yields the existence of a globally defined section $t \in \mc{F}(U)$ such that $t|_{U_x} = s|_x$ for all $x$. 

It remains to show that $s = t$. This follows from another application of the lemma: by construction, they $s_x = t_x$ for all $x \in U$, so by the lemma, $s = t$ as a section in $\mc{F}(U)$.

\qed

Recall from friday the claim:

\claim

Let $f:Y\to X$ be continuous. Then $f_*:\Psh(Y)\to\Psh(X)$ sends $f_*(\Sh(Y)) \subset \Sh(X)$, where the inclusion means the essential image.

\proof

We just have to check that the pushforward $f_*\mc{F}$ also satisfies the gluing condition. 

By definition
\[
f_*\mc{F} \eqdef \mc{F}\circ f^{-1}
\]
This is the composition of $f^{-1}:\operatorname{Open}(X)^{op}\to \operatorname{Open}(Y)$ and $\mc{F}:\operatorname{Open}(Y)\to\Set$.

And the preimage of an open cover is an open cover.

\qed

\subsection*{\underline{Locally ringed spaces}}

\defn

Let $X$ be a space. A \underline{ring object} is an object $\mc{R}$ along with two ``binary operations," $+,\cdot :\mc{R}\times\mc{R}\to\mc{R}$, and maps (thought of as sections) $0,1:X\to\mc{R}$ (i.e. $0,1 \in \mc{R}(X)$), such that the usual ring axioms, re-expressed by commutative diagrams, hold. 

Commutativity of addition:
\[
\begin{tikzcd}
\mc{R}\times\mc{R}\ar[drr,"+"] \ar[rr, "swap"] & & \mc{R}\times\mc{R} \ar[d,"+"] \\
& & \mc{R}
\end{tikzcd}
\]
Existence of identity:
\[
\begin{tikzcd}
\mc{R} \ar[r,"\Id_R\times c_1"] \ar[d, "\Id_R"] & \mc{R}\times\{1\} \ar[d, "\Id\times c"] \\
\mc{R} & \ar[l,"\cdot"'] \mc{R}\times\mc{R}
\end{tikzcd}
\]

et cetera. 

\defn

Let $X$ be a space. Then a \underline{ring object} $\mc{R}$ in the category $\Sh(X)$ is called a \underline{sheaf of rings} on $X$. The pair $(X,\mc{R})$ is called a \underline{ringed space}. 

This is equivalent to a presheaf $\mc{R}:\operatorname{Open}(X)^{op} \to \Ring$ with the gluing condition. 

\section*{Lecture 6, 17/9/25}

\underline{Today:} Locally ringed spaces

\underline{Last time:} Ringed spaces.

Recall a ring space is a pair $(X,\mc{R})$, consisting of a topological space $X$, and a ring object $\mc{R}$ in $\Sh(X)$. 

\exm
\,
\begin{itemize} 

\item Let $X = \{*\}$. Then $\mc{R}$ is just a ring. This is because $\Sh(X)$ in that case is equivalent to $\Set$, and a ring object in $\Set$ is just a ring. 

\item Take $X$ to be any space. We can define the sheaf $C(X)$ of continuous real-valued functions. So as a presheaf, $C(X)(U) = \Hom(U,\R)$. 

\end{itemize}

\defn Given a ring object $\mc{R}$, we define $\mc{R}^*$ as the fiber product
\[
\begin{tikzcd}
\mc{R}^* \ar[r] \ar[d] & \{1\} \ar[d]\\
\mc{R}\times\mc{R} \ar[r, "\cdot"] & \mc{R}
\end{tikzcd}
\]
So $\mc{R}^* = (\mc{R}\times\mc{R})\times_{\mc{R}}\{1\}$. This is the set of pairs of elements of $\mc{R}$ which multiply to 1. The projections being injective mean that we can view it as the set of invertible elements of $\mc{R}$.

\claim

The projections from $\mc{R}^*$ to $\mc{R}$ are injections. That is, $\mc{R}^* \subset \mc{R}$ is a subsheaf. 

So the inclusion is an open map of \'etal\'e spaces. 

\defn 

Given $(X,\mc{R})$, $U \subset X$ open, $f \in \mc{R}(U)$, we define $U_f$ by 
\[
\begin{tikzcd}
U_f \ar[r] \ar[d] & \mc{R}^* \ar[d] \\
U \ar[r, "f"] & \mc{R}
\end{tikzcd}
\]

So $U_f = f^{-1}(\mc{R}^*) = \{x \in U \mid f(x) \in \mc{R}^*\}$

\claim 

$\mc{R}^*$ is a group object in $\Sh(X)$. 

\proof

Volunteer

\qed

\defn

Let $\mc{R} \in \Sh(X)$ be a ring object. Then $\mc{R}$ is \underline{local} if for all open $U \subset X$, for all $f \in \mc{R}(U)$, we have $U = U_f \cup U_{1-f}$.

This means that for every point $x \in U$, at least one of $f(x), 1 - f(x)$ are invertible. 

In this case, we say that $(X,\mc{R})$ is a \underline{Locally ringed space}. 
\exm
\,
\begin{itemize}

\item $(X, C_X)$ is local 

\item $(\{*\},\mc{R})$ is local if and only if there is a unique maximal ideal in $\mc{R}$. 

\end{itemize}

\lem

Let $(X,\mc{R})$ be a locally ringed space, $U \subset X$ open, and $f_1, \dots, f_r \in \mc{R}(U)$. Then
\[
f_1 + \cdots + f_r = 1 \implies \bigcup_{i=1}^r U_f = U
\]

\proof

Volunteer

\qed

\underline{Goal:} We want to associate to a ring $R$ a \underline{universal} locally ringed space $(X,\mc{R})$, such that  there exists a ring homomorphism
\[
R \to \mc{R}(X)
\]
with the property that, given a locally ringed space $(Y,\mc{S})$, with a homomorphism of rings $R\to \mc{S}(Y)$, we want a unique map of locally ringed spaces from $(Y,\mc{S})$ to $(X,\mc{R})$, such that ...

The solution $(X,\mc{R})$ will be denoted by $(\Spec R, \mc{O})$

This is the same notation used for the set of prime ideals. \underline{Why}?

Assume that such a universal locally ringed space exists. 

$|X|$ denotes the underlying set or space of points of a locally ringd space $(X,\mc{R})$

\claim

$|X|$ is in canonical bijection with the set of prime ideals of $\R$.
\[
|X| \cong \{\mk{p} \subset R \mid \mk{p}\text{ is prime} \}
\]

\proof

To go from left to right, we take the ring $R$, and localize $R_{\mk{p}}$, meaning we invert every element in the complement of $\mk{p}$. This is by design a local ring, and we can think of this as being a locally ringed space $(\{*\}, R_{\mk{p}})$. By universality we get a map into $(X,\mc{R})$, and we consider the image of the point $\{*\}$. 

To go the other way, we consider $\mc{R}_x$. This will be a local ring, and there is a map of rings frome $R \to \mc{R}_x$. $\mc{R}_x$ being a local ring means there is a unique maximal ideal $\mk{m}$. We take this maximal ideal, and take its preimage under this ring homomorphism. The preimage of a maximal ideal may not be maximal, but the preimage of a prime ideal is always prime, and maximal ideals are prime. So the preimage is prime.

\qed

\underline{Candidate topology} on $\Spec \mc{R}$ is the smallest topology containing $U_f = \{ \mk{p} \in \Spec R \mid f \not\in\mk{p}\}$. 

This is called the Zariski topology on $|\Spec R|$. They key thing is that these subsets are always open in ringed spaces. 
 
\defn[Structure Sheaf $\mc{O}$ on $|\Spec R|$]

Let $\mc{O} \eqdef \coprod_{\mk{p}\in|\Spec R|}R_{\mk{p}}$, with the topology generated by the sections $\frac{g}{f^n}:U_f \to \mc{O}$ for all $\frac{g}{f^n} \in R_f$.

\section*{Lecture 7, 19/9/25}

First the volunteers prove the things promised last time. 

\claim 

Let $(X,\mc{R})$ be a locally ringed space, $U \subseteq X$ open, $f_1, \dots, f_n \in \mc{R}(U)$. Then
\[
\sum_{i=1}^r f_i = 1 \implies U = \bigcup_{i=1}^r U_{f_i}
\]

\proof

We have a map $\mc{R}(U) \to \mc{R}_x$ by definition of a colimit. Because we are in a locally ringed space, it has a maximal ideal, $(\mc{R}_x,\mk{m}_x)$. We can (equivalently to what we said earlier) define $U_{f_i} = \{x \in X \mid \barf\not\equiv0\text{ mod } \mk{m}_x\}$. 

Geometrically, this means $\{x\in U \mid \barf(x) \neq 0\}$. 

By assumption, at least one of the $f_i \in \mk{m}_x$. Then $\sum_{i=1}^rf_i\in\mk{m}_x$.

Thus one of the $f_i$ is a unit. By ring theory, units map to units, so every $x$ lies in a $U_{f_i}$. 

\qed

\claim 

$\mc{R}^*$ is a group object in the category $\Sh(X)$. 

\proof

Recall

We have the diagram 
\[
\begin{tikzcd}
\ar[d] \mc{R}^* \ar[r] & \{1\} \ar[d] \\
\mc{R}\times\mc{R} \ar[r] & \mc{R}
\end{tikzcd}
\]

Then we have $\pi_1:\mc{R}^*\into \mc{R}$, this is injective on stalks $\mc{R}_x^* \to$ units in $\mc{R}_x$ (we are identifying $\mc{R}^*$ with its image). 

So we have an injective mapping $\mc{R}^*(U) \into \mc{R}(U)^*$. This is onto since $r \in \mc{R}(U)^*$ implies $(r, r^{-1}) \in \mc{R}^*(U)$. 

If $s, t \in \mc{R}^*(U_x)$, then $s_x, t_x \in \mc{R}^*_x$ for any $x \in U$, so we get $U_x \supset \{x\}$ and $s', t' \in \mc{R}(U_x)^*$, so $U = \bigcup_{x}U_x$. 

So $\mc{R}(U)^*$ contains $s|_{U_x} = s_x$. 

I didn't really follow this sorry, I should think more about this / ask the guy about his proof. 

\qed

\underline{Slogan:} If $(X,\mc{R})$, $f \in \mc{R}(U)$ is a locally ringed space, then for $x \in X$, $\barf_x \in \mc{R}_x/\mk{m}_x = \kappa(x)$ can be thought of as the value of the ``generalized function" $f$ with values $\barf_x \in \kappa(x)$ at all points $x$. 

Watch out: thinking of this as a function taking values is risky, because there is a choice involved. 

\underline{$\Spec R$}

Let $R$ be a ring. Then we have a locally ringed space $(\Spec R, \mc{O})$

\underline{Topological space:} We define $|\Spec R| = \{\mk{p} \mid \mk{p}\subset R\text{ is prime } \}$, endowed with the Zariski topology, i.e. the topology generated by open subsets $U_f \eqdef \{\mk{p} \in |\Spec R| \mid f \not\in \mk{p}\}$, for any $f \in R$. 

\underline{Motivation:} Historically speaking, this is how we define the topology on $\Specmax$. 

More importantly, we have this universality, since for $(X,\mc{R})$ locally ringed and $f \in \Gamma(U,\mc{R}), U_f$ is open. 

\subsection*{\underline{Structure sheaf $\mc{O}$}}

As a set, the \'etal\'e space, 
\[
\mc{O} = \coprod_{\mk{p}\in|\Spec R|} R_{\mk{p}}
\]

The \'etal\'e projection sends $R_{\mk{p}}$ to $\mk{p}$. 

The topology is generated as follows. For any $\frac{g}{f^n} = s \in R_f = R[f^{-1}]$, and any point $\mk{p}$ we get a map sending $s$ to its image in $R_{\mk{p}}$ (because $f$ is not in $\mk{p}$ by def of $U_f$). The topology is generated by the images of all of these. 

The topology on $\mc{O}$ is the universal one, such that a map $g:\mc{O}\to Y$ is continuous if and only if $g\circ s$ is continuous for all $f$, for all $s$ as above. 

This is an \'etal\'e space by construction, such that the stalks are the local rings $R_{\mk{p}}$. That is, 
\[
\mc{O}_{\mk{p}} = R_{\mk{p}}
\]

are \underline{local rings}.

Thus, $(|\Spec R|, \mc{O})$ is a locally ringed space, where $\mc{O}(U_f) = R_f$. The advantage over the typical definition of a sheaf is that we have only defined the sections on a subbasis, and we have to prove some things about how we can sheafify these into a sheaf. 

\thm Let $X = (|X|,\mc{R})$ be a locally ringed space, and $\varphi:R\to\mc{R}(X)$ be a ring homomorphism. Then 
\begin{enumerate}[label=(\alph*)]

\item There exists a morphism of locally ringed spaces $X \to \Spec R$

\item This map is universal with respect to this property. 

\end{enumerate}

\proof

Next time

\qed

\defn

A \underline{morphism of locally ringed spaces} $f:Y\to X$ is a continuous map $|f|:|Y|\to|X|$, along with a morphism of sheaves of rings (a morphism of ring objects in $\Sh(X)$) $f^\sharp: f^{-1}(\mc{R}_X)\to\mc{R}_Y$ such that for all $y \in Y$, the map between stalks 
\[
\mc{R}_{X,|f|(y)}\to\mc{R}_{Y,y}
\]
is a \underline{local ring map}, i.e. $(f^\sharp_y)^{-1}(\mk{m}_y)= \mk{m}_{|f|(y)}$

\section*{Lecture 8, 22/9/25}

\underline{Today:} Prove the universal property of $\Spec R$, global sections on $\Spec R$ of the structure sheaf $\mc{O}$, (affine) schemes. 

\underline{Recall:} Given a ring $R$, there exists a topological space $|\Spec R|$, whose points are the prime ideals of $R$, and endowed with the Zariski topology, whose opens are generated by sets of the form $D_f = \{\mk{p} \in \Spec R \mid f \not\in \mk{p}\}$

\underline{Reading assignment:} Finish 2.2 in Hartshorne by the end of this week. 

\paragraph{}

On the space $|\Spec R|$, we have the structure sheaf $\mc{O}_R$, constructed by putting a topology on $\coprod_{\mk{p}\in|\Spec R|}R_{\mk{p}}$,  generated as follows:

For every $s \in R_f$, we have a function from $D(f)$ to $\mc{O}_R$ given by sending a prime ideal $\mk{p}$ to the image of $s$ in $R_{\mk{p}}$, and by construction this is both well defined and a section of the map $\mc{O}_R\to |\Spec R|$. 

We want the coarses topology making these maps continuous. 

We refer to the pair $(|\Spec R|, \mc{O}_R)$ by simply $\Spec R$. 

\lem

Let $X = (|X|, \mc{O}_X)$ be a locally ringed space and $\varphi:R\to\mc{O}_X(|X|)$ a ring homomorphism. Then the following map $\Phi$ from $|X|$ to $|\Spec R|$, given by $x\mapsto \varphi^{-1}(\mk{m}_x) \in |\Spec R|$ is continuous. 

\proof

\underline{Remark:} For all $x \in |X|$, we have a local ring $(\mc{O}_X)_x$, which contains a unique maximal ideal, which we denote by $\mk{m}_x$ (or $\mk{m}_{X,x}$), and a ring homomorphism $\varphi:R\to\mc{O}_{X,x}$, which we can pull back to get a prime ideal in $R$ (this map is defined because the original map $\varphi$ goes to $\mc{O}_X(|X|)$, which is a set of sections, and each of those sections has an image in $\mc{O}_{X,x}$). Now for the proof:
\paragraph{}
By definition, it suffices to check, for all $f \in R$, that $\Phi^{-1}\left(D(f)\right) \subset |X|$ is \underline{open} in $|X|$. 

\claim
\[
\Phi^{-1}\left(D(f)\right) = |X|_{\varphi(f)} = \{x \in X \mid \varphi(f)_x \in \mc{O}_{X,x}^\times = \mc{O}_{X,x}\setminus\mk{m}_x\}
\]
\proof

By definition of $\Phi$, we have ring homomorphisms 
\[
\begin{tikzcd}
R \ar[r,"\phi"] \ar[dr,"q"] & \mc{O}_{X,x} \\
& R_{\mk{p}} = \Phi(x) \ar[u, "\exists!", dotted] 
\end{tikzcd}
\]

This is a local ring at $\mk{p}$, and it detects and reflects invertibility. 

So $\Phi^{-1}\left(D(f)\right) = |X|_{\varphi(f)}$

\qed

\lem

$\Phi$ can be promoted to a morphism of locally ringed spaces: $X\to \Spec R$.

\proof

We need to build a map of sheaves of rings $\Phi^{-1}\mc{O}_R \to \mc{O}_X$

We will do this stalk by stalk: For 
\[
\begin{tikzcd}
\coprod_{x\in|X|}R_{\mk{p} = \Phi(x)} \ar[d] \ar[r,"\varphi_x"] & \coprod \mc{O}_{X,x} \ar[d]\\
X & X \\
\end{tikzcd}
\]
Where the top right is the topology given by using the sheafification viewing it as a presheaf of sections. 

We have the map $\varphi_x:R_{\mk{p}}\to \mc{O}_{X,x}$, $\mk{p} = \varphi^{-1}(\mk{m}_x)$ is a local ring map, so $(\Phi, (\varphi_x)_{x\in X})$ is a local ring map, granting continuity of $\to$. 

This uses the definition of the topology on the \'etal\'e space of $\mc{O}_R$. 

\qed

\lem 

Given $(|X|, \mc{O}_X), (|Y|,\mc{O}_Y)$ with ring homomorphisms $\varphi:R \to \mc{O}_X(|X|), \psi:R \to \mc{O}_Y(|Y|)$ and a morphism of locally ringed spaces $g:Y\to X$ which is compatible with $\varphi$ and $\psi$, then there is a commutative diagram in the category of locally ringed spaces
\[
\begin{tikzcd}
Y \ar[r,"\Psi"] \ar[dr, "g"'] & \Spec R \\
& X \ar[u, "\Phi"']
\end{tikzcd}
\]
where $\Psi, \Phi$ are the maps induced by $\psi, \varphi$, as in the previous lemma. 

\proof

\qed

This construction yields a functor $\Spec:\Ring^{op} \to \operatorname{LocRingdSp}$

\thm

The functor $\Spec$ is an embedding and a (quasi)-inverse (on the essential image of $\Spec$) is given by taking global sections $\Gamma(\mc{O}):\operatorname{LocRingdSp}^{op}\to\Ring$, $(|X|,\mc{O}_X) \mapsto \mc{O}_X(|X|)$.

\proof

In class presentation. 

\qed

The key lemma is the following:

\lem

$\mc{O}_R(|\Spec R|) \cong R$

\proof

In class presentation

\qed

With all this, the following definitions finally make sense:
\,
\defn
\,
\begin{enumerate}

\item A locally ringed space $X$, isomorphic to $\Spec R$ for some $R$, is called an \underline{affine scheme}

\item A \underline{scheme} is a locally ringed space $X = (|X|, \mc{O}_X)$ such that there exists an open cover $\bigcup|U_i| = |X|$, such that for all $i$, we have that $U_i = (|U_i|, \mc{O}_X|_{U_i})$ is an \underline{affine scheme}.

\end{enumerate}

\underline{Analogy:} 

Recall that a smooth manifold is a topological space with certain properties (2nd countable, hausdorff) with a set of smooth charts. If we forget the desired topological spaces, then this category also embeds into the category of locally ringed spaces, with the functor given by sending $M = (|M|, \mc{C}^\oo_M)$, where $\mc{U}$ is a (maximal) atlas to $(|M|, \mc{O}_M)$, where $\mc{O}_M(U)$ is the set of $\R$-valued smooth functions on $U$. 

So $(|M|,\mc{C}^\oo_M)$ is covered by sets that look like $(\R^n, \mc{C}^\oo_{\R^n})$

\section*{Lecture 9, 24/9/25}

\underline{Summary:} An object of $\operatorname{LocRingdSpc}$ is a pair $(|X|,\mc{O}) = X$, with $|X|$ a space, $\mc{O}$ a ring object in $\Sh(|X|)$ which is \underline{local}, which means that for any $u$, and any section $s \in \mc{O}(U)$, $U_s \cup U_{1-s} = U$, where $U_s$ is defined by the fiber product
\[
\begin{tikzcd}
U_s \ar[r] \ar[d] & \mc{O}^* \ar[d]\\
U \ar[r,"s"] & \mc{O}
\end{tikzcd}
\]
Equivalently, $U_s = \{x\in U \mid s(x) \in \mc{O}^*\}$.
So, given any two sections $s, t$ with $s + t = 1$, their loci of invertibility cover $U$, in analogy with the weak nullstellensatz. This implies that all stalks are local rings, that is $\mc{O}_x$ has a unique maximal ideal, $\mk{m}_x$.

\underline{Last time:} We have a bijection 
\[
\Hom_{\LRS}(X,\Spec R) = \Hom_{\Ring}(R,\underbrace{\mc{O}_X(|X|)}_{\text{ring of generalized functions on $X$}})
\]

\underline{Stated:} $\Gamma(|\Spec R|,\mc{O}_R) \cong R$

(to be shown on friday)

\underline{Today:} Examples of schemes

\underline{Affine Line:} $\mbb{A}_{\Z}^n \eqdef \Spec \Z[T_1,\dots, T_n]$

\cor 

$\mc{O}_X(|X|) = \Hom_{\LRS}(X, \mbb{A}^n_\Z)$

\proof
\,

Consider $\varphi \in \Hom_{\Ring}(\Z[T], \mc{O}_X(|X|))$, and map it to $\varphi(T) \in \mc{O}_X(|X|)$.

So by the theorem stated above, then $\mc{O}_X(|X|) = \Hom(X,\mbb{A}_\Z^n)$

So generalized functions on $X$ are in 1-1 correspondence with morphisms $X \to \mbb{A}_\Z^n$

\qed

\underline{Remark:} $\mbb{A}_\Z^1$ is a \underline{ring object} in $\LRS$. This is very deep, but also tautological. Abstract nonsense...


\defn

$\mbb{G}_{m,\Z} \eqdef \Spec \Z[T,T^{-1}]$, often called the \underline{multiplicative group (scheme)}. 

\cor

$\Hom(X,\mbb{G}_{m,\Z} = \mc{O}_X(|X|)^*$. 

\defn

$\GL_{n,\Z} = \Spec \Z[T_{11},\dots,T_{nn}, \det\left((T_{ij})\right)^{-1}]$

\cor

$\Hom(X,\GL_{n,\Z}) = \{A = (s_{ij})_{i,j = n}\mid \det A$ multiple of $s_{ij} \in \mc{O}_X(|X|)\}$ (???)

\proof

\qed

\underline{Remark:} $\mbb{G}_{m,\Z}, \GL_{n,\Z}$ are \underline{group objects} in $\LRS$.

\subsection*{\underline{Fiber Products I}}

\lem

Suppose we have ring homomorphisms
\[
\begin{tikzcd}
B\ar[r] \ar[d] & C \\
A & 
\end{tikzcd}
\]
Then $\Spec(A\otimes_BC)$ is the fiber product $\Spec A\times_{\Spec B}\Spec C$ in $\LRS$.

\proof

Consider the set of maps $\{A\otimes_BC \to \mc{O}_X(|X|)\}$, whic h we identify with $\Hom(X,\Spec (A\otimes_BC))$. But the first is equal to the fiber product
\[
\begin{tikzcd}
B \ar[r] \ar[d] & C \ar[d, "\psi"] \\
A \ar[r, "\varphi"] & \mc{O}_X(|X|)
\end{tikzcd}
\]
, where we send $a\otimes c$ to $\varphi(a)\cdot\psi(b)$. But by the same property this is the same as the coproduct 
\[
\begin{tikzcd}
X \ar[r] \ar[d] & \Spec C\ar[d] \\
\Spec A \ar[r] &\Spec B
\end{tikzcd}
\]
So this is establishes the universal property, 
\[
\begin{tikzcd}
X \ar[drr, bend left = 20] \ar[ddr, bend right = 20] \ar[dr,dotted,"\exists!"] & & \\
& \Spec (A\otimes_BC) \ar[r] \ar[d] & \Spec C \\
& \Spec A \ar[r] & \Spec B
\end{tikzcd}
\]
\qed

\underline{Remark:}

We have the correspondence between
\[
\begin{tikzcd}
X \ar[r] \ar[dr] & \Spec A \ar[d] \\
& \Spec B
\end{tikzcd}
\]
and
\[
\begin{tikzcd}
A \ar[r] & \mc{O}_X(|X|) \\
B \ar[u]\ar[ur] &
\end{tikzcd}
\]

\underline{Concrete examples}

\begin{enumerate}

\item $\varnothing = \bigcup \varnothing$, and it is a scheme, because $\varnothing = \Spec 0$, which is an affine scheme.

\item Let $k$ be a field. Then $|\Spec k| = \{*\}$, the zero ideal in $k$. Then $\Spec k = (\{*\}, k)$, a point with a field attached to it. 

\item $\Spec k[\varepsilon]/(\varepsilon^2)$ 

\item The spectrum of any artinian local ring. In this and the above case, the set of prime ideals is a single point. 
\underline{Cor:} 

Take $R = \Z[S,T]/(S^2 - T^2 - 1)$, and $X = \Spec R$. Then
\[
\Hom(\Spec \R, X) = \{(x,y) \in \R^2 \mid x^2 + y^2 = 1\}
\]
and
\[
\Hom(\Spec\R[\varepsilon]/(\varepsilon^2), X) = \{(z,v) \in \R^2\times \R^2 \mid z \in S^1, v \perp z\} = TS^1
\]

More generally, 
\[
\Hom(\Spec k, X) = k\text{-solutions to a system of equations}
\]
\[
\Hom(\Spec k[\varepsilon]/(\varepsilon^2), X) = k\text{-tangent vectors at $k$-points of a variety}
\]

\item $\Spec\Z$ is final in $\LRS$, i.e. for all $X$, there is a unique map $X \to \Spec \Z$, because such a map corresponds to a ring homomorphism $\Z \to \mc{O}_X(|X|)$, and $\Z$ is initial in $\Ring$ (because $1$ must be sent to $1$). 

Observe $|\Spec \Z| = \{(0)\} \cup \{\text{prime numbers}\}$, with topology generated by $U_n = \{(0), p \mid p\nmid n\}$, $n \in \Z$. All non-empty opens contain $(0)$. So $U_n\setminus\{(0)\} = \mbb{P}\setminus(\text{finite subset})$. So it is basically the cofinite topology, except there is an additional point in every open subset. So $\bar{(O)} = |\Spec\Z|$. We will refer to such points as \underline{generic points}. 

If we consider $\mc{O}_\Z(|\Spec\Z|) = \Z$, then $\mc{O}_{\Z,p} = \Z[q^{-1} \mid q \neq p]$, and $\mc{O}_{\Z,(0)} = \Z_{(0)} = \Q$. 

To see this, consider $s \in \Gamma(|\Spec \Z|)$. This is Zariski locally represented by a fraction $s_i \in U_{d_i}, s_i = \frac{n_i}{d_i}$. Then by locality $\cup_i U_{d_i} = |\Spec \Z|$ is equivalent to $\gcd(d_i) = 1$. Since $\gcd(d_i) = 1$ and $\frac{n_i}{d_i} = \frac{n_j}{d_j}$ for each $i, j$, this implies $\frac{n_i}{d_i} = \lambda\in\Z$. This works in any $PID$. 

\end{enumerate}

\section*{10, 26/9/25}

\underline{Presentation: Kareem}

Facts to take for granted: $V(I) = \{p \in |\Spec R\ \mid I \subseteq p\}$, $D(f)^c = V\left((f)\right)$, $V(I) = V(\sqrt{I}) \subset V(J) \subset V(\sqrt{j})$ iff $\sqrt{I} \supset \sqrt{j}$

$\bigcap V(I) = V(\sum I)$. 

\claim

$\varphi: R_f \to \mc{O}(D(f))$ is an isomorphism, with the map sending $\frac{a}{f^n}$ to $s(\mk{p}) = \frac{a}{f^n}$ in $V_\mk{p}$. 

\proof

Kareem went too fast. Sorry

\qed

\underline{Fact:} $|\Spec R|$ is always quasi-compact, no matter the ring, so the underlying space of an affine scheme is always quasi-compact. 

So for any family of non-zero $R_i$, $\sqcup_{n\in\Z}|\Spec R_n|$ is a non-affine scheme. It is not affine because the underlying space is not quasicompact. 

\underline{Remark:} It is true that if you take the finite product of rings, then 
\[
\Spec \left(\prod_{i=1}^r R_i\right) = \sqcup_{i=1}^r \Spec R_i
\]
But this is not true for infinite products, because $\Spec\left(\prod_{i\in I} R_i\right)$ is quasicompact. 

Contemplate $|\Spec \mbb{F}_2^\N|$.

\subsection*{\underline{Criterion for affine-ness}}

Let $X = (|X|,\mc{O})$ be a locally ringed space. Then there is a unique homomorphism to an affine scheme $\aff_X:X \to \Spec \O(|X|)$ called the \underline{affinization of $X$}, denoted $\operatorname{aff}_X$

\underline{And:}

$X$ is affine if and only if $\aff_X$ is an isomorphism of locally ringed spaces. 

\underline{$\mbb{A}_\Z^n$ with a doubled origin}

\underline{Recall:} $\mbb{A}_\Z^1  \eqdef \Spec \Z[T]$. This contains $D(T) = |\Spec \Z[T,T^{-1}]| = |\mbb{G}_{m,\Z}|$. 

\defn

We denote by 
\[
\mbb{A}_{\Z,double-O}^1 \eqdef \mbb{A}_\Z^1 \sqcup_{\mbb{G}_{m,\Z}} \mbb{A}_\Z^1
\]
This pushout still exists in the category of locally ringed spaces, so this definition makes sense. Indeed, the underlying topological space is this pushout. 

We define the structure sheaf to be the unique $\O$ such that $\O|_{\mbb{A}_\Z^1} = \O_{\mbb{A}_\Z^1}$, where those are the two copies of $\mbb{A}_{\Z}^1$. 

By construction, $X$ is a scheme, because it can be covered by two affine scheme. 

\claim

$X$ is not affine. 

\proof
\,
\begin{enumerate}

\item First, consider the equalizer of the following diagram:
\[
\begin{tikzcd}
\O_X(|X|) \ar[r]& \O_{\mbb{A}^n}(\mbb{A}^n) \oplus \O(\mbb{A}^n) \ar[r, bend left = 20] \ar[r, bend right = 20]& \O(\mbb{G}_m)
\end{tikzcd}
\]
Where the top map is $T_1 \mapsto T$, and the bottom $T_2 \mapsto T$, where $T_i$ are the two copies of $T$. Recall $\O_X(|X|) = \Z[T]$, so this diagram reads 
\[
\begin{tikzcd}
\Z[T] \ar[r] & \Z[T_1] \oplus \Z[T_2] \ar[r, "T_1\mapsto T", bend left = 20] \ar[r, "T_2 \mapsto T", bend right = 20] & \Z[T, T^{-1}]
\end{tikzcd}
\]
But $\aff_X:X\to \mbb{A}_\Z^1$ on $|\mbb{G}_{m,\Z}|$ $\aff_X|_{\mbb{G}_m}$ is not an iso. Indeed, 
\[
|\aff_X^{-1}(\mk{m}_0)| = 2
\]
Because $\aff_X^{-1}(|\mbb{G}|) = |\mbb{G}|$, so $|X\setminus \aff_X^{-1}(|\mbb{G}|)| = 2$, thus $|\aff_X|$ is not injective, so $\aff_X$ cannot be an isomorphism of locally ringed spaces. 
\end{enumerate}

\qed

\underline{Another example:} $\mbb{P}_\Z^1$. 

Consider again the pushout of spaces 
\[
\begin{tikzcd}
\mbb{G}_{m,\Z} \ar[r, hook] \ar[d, hook] & \mbb{A}_-^n \ar[d]\\
\mbb{A}_+^1\ar[r, hook] & \mbb{P}_\Z^1
\end{tikzcd}
\]
Where $\mbb{A}_\pm^1 = \Spec \Z[T^\pm]$, and of course $\mbb{G}_{m,\Z} = \Spec \Z[T,T^{-1}]$. 

We define this by first taking the pushout of topological spaces to the topological space, and then glue the structure sheaves together on $|\mbb{A}_+^1|,|\mbb{A}_-^1|$ accordingly. We call this $\O_{\mbb{P}^1_\Z}$. This is by construction covered by two affine schemes and thus a scheme. 

To comput $\O_{\mbb{P}_\Z^1}(|\mbb{P}_\Z^1|)$, we look at another equalizer
\[
\begin{tikzcd}
\O_{\mbb{P}_\Z^1}(|\mbb{P}_\Z^1|) \ar[r] &\mc{O}_{\mbb{A}_+^1}(|\mbb{A}_+^1|) \oplus \O_{\mbb{A}_-^1}(|\mbb{A}_-^1|) \ar[r, bend left = 20] \ar[r, bend right = 20] &\Gamma(\O_{\mbb{P}_\Z^1})
\end{tikzcd}
\]
Agai, this diagram reads 
\[
\begin{tikzcd}
\Z[T] \cap \Z[T^{-1}] \ar[r] & \Z[T] \oplus \Z[T^{-1}] \ar[r, bend left = 20], \ar[r, bend right = 20] & \Z[T,T^{-1}]
\end{tikzcd}
\]

So $\aff_{\mbb{P}_\Z^1}:\mbb{P}_\Z^1 \to \Spec \Z$, but $|\mbb{P}_\Z^1| \supset |\mbb{A}_\Z^1| $ is sent to $|\Spec \Z|$, so $\aff$ cannot be an isomorphism here. 

\section*{Lecture 11, 29/9/25}

\underline{Reading assignment:} Section 2.3 (start with definition of an open subscheme).

\underline{Next quiz:} Friday

\underline{Today:} 
\begin{itemize}

\item Open immersions

\item Fibre products

\end{itemize}

\defn

A morphism of locally ringed spaces $(|f|, f_*) = f:Y\to X$ is called an \underline{open immersion} if 
\begin{itemize}

\item $|f|:|Y|\to|X|$ is \underline{open} and \underline{injective}

\item $f_*:f^{-1}\O_x \to \mc{O}_Y$ is an \underline{isomorphism} of sheaves

\end{itemize}

\defn

Let $X$ be a locally ringed space, $|U|\subseteq |X|$ an open subset. Then we define 
\[
U \eqdef (|U|, \O_X|_{|U|})
\]

This is also a locally ringed space, and the map $\iota:U\into X$ (given by the inclusion, and the ring isomorphism $\iota_*:\O_X|_{|U|} \to \O_U$) is an \underline{open immersion}

\underline{Fact:}

For an open immersion $f:Y\to X$, there exists an open subset $|U|\subset |X|$ such that there is a commutative diagram
\[
\begin{tikzcd}
Y \ar[dr, "f"] & \\
& X \\
U \ar[uu, "\simeq"] \ar[ur, "\iota_U"'] &
\end{tikzcd}
\]

\underline{Remark:} A morphism of locally ringed spaces $f:Y\to X$ which is open and injective is an open immersion if and only if for all $y \in Y$, the map $\O_{Y,y} \to \O_{X,f(y)}$ is an \underline{isomorphism}

\defn

If $X$ is a scheme and $|U|\subset |X|$ is open, then $U$ is called an \underline{open subscheme} of $X$. 

\lem

$U = (|U|, \O_X|_U)$ is a scheme if $X$ is a scheme. 

\proof

It suffices to show this for $X = \Spec R$. 

Let $|U| \subset |\Spec R|$ be open. Then $|U| = \bigcup_{\alpha\in A} D(f_\alpha)$. Note $D(f_\alpha) = |\Spec R_{f_\alpha}|$.

So $U$ is covered by the affine schemes $\Spec R_{f_\alpha}$

\qed

\underline{Warning:} $U$ is not necessarily affine. 

\exm

Recall $\mbb{A}_\Z^2 = \Spec\Z[S,T]$. Consider $|U| = D(S) \cup D(T)$

Note $|U| \neq |\mbb{A}_\Z^2|$, and in fact the complement $|\mbb{A}_\Z^2|\setminus|U|$ is generated by $V\left((S,T)\right) = |\Spec \Z|$. Then
\[
\Z[S,T] = \O_U(|U|) = \O_{D(S)}(D(S)) \cap \O_{D(T)}(D(T)) = \Z[S,T]_S \cap \Z[S,T]_T
\]

So $\aff_U:U\to \mbb{A}_\Z^2$ (given by $\iota)$ is not surjective. So $U$ is non-affine.

\lem

Let 
\[
\begin{tikzcd}
& Z \ar[d, "g"] \\
X \ar[r, "f"] & Y \\
\end{tikzcd}
\]
be a diagram of locally ringed spaces, with $f$ and \underline{open immersion}. 

Then $X\times_YZ$ exists, and the morphism $f':X\times_YZ\to Z$ given by projection onto the second coordinate is an \underline{open immersion}

So we can lift open immersions

\proof

Consider $\left(\underbrace{ g^{-1}(f(|X|)}_{|U|}),\O_Z|_{|U|}\right)$, and show that it has the required properties and satisfies the universal property of the fiber product. 

First, $(g^{-1}(|U|), \O_Z|_{g^{-1}(|U|)}) \to (|Z|,\O_Z)$ given by the inclusion of $|g^{-1}(U)|$ is an open immersion. 

Let $W \in \LRS$. Then consider the diagram
\[
\begin{tikzcd}
W \ar[drr, bend left = 20] \ar[ddr, bend right = 20] \ar[dr,"\exists!", dotted] & & \\
& g^{-1}(|U|, \O_Z|_{g^{-1}(|U|)})\ar[d] \ar[r] & Z \ar[d,"g"']\\
& U \ar[r, "f"] & Y
\end{tikzcd}
\]

The dotted map exists and is unique just on the level of continuous maps automatically, $h:W\to Z$ corresponds to $h^{-1}\O_Z \to \O_W$, where $\O_Z = h^{-1}(\O_Z|_{g^{-1}(|U|)})$, recalling that $f$ is an open immersion. 

\qed

\exm

For $U, V \subset X$ are subschemes, then for the cartesian diagram
\[
\begin{tikzcd}
U\cap V \ar[r, hook] \ar[d, hook] & V \ar[d, hook] \\
U \ar[r, hook] & X
\end{tikzcd}
\]
We haved $U\cap V = U\times_XV$

\underline{Pushouts along open immersions}

\lem

Let 
\[
\begin{tikzcd}
X \ar[r,"f"] \ar[d,"g"'] & Y \\
Z & 
\end{tikzcd}
\]
be a diagram of schemes and open immersions. Then the pushout $X \sqcup_Y Z$ exists in the category of schemes. 

\proof

Define $|Y\sqcup_XZ|$ by taking the pushout in the category of topological spaces, i.e. $\frac{|Y|\sqcup|Z|}{\sim}$, where $\sim$ identifies $f(x)$ and $g(x)$ for all $x \in |X|$. 

\underline{Topology:} $U \subset |Y| \sqcup_{|X|}|Z|$ is open if and only if $U \cap |Y|$ and $U \cap |Z|$ are open. 

Glue the \'etal\'e spaces of $\O_Y,\O_Z$ along $\O_X$ the same way. 

This shows its a locally ringed space because the stalks are local rings, and is also covered by affine schemes. 

\qed

\lem

Assume that in the diagram below, both inner squares are \underline{Cartesian} (meaning fibre products).
\[
\begin{tikzcd}
W' \ar[d] \ar[r] \ar[dr,phantom]& W \ar[d] \ar[r] & Z \ar[d]\\
X'\ar[r] & X \ar[r] & Y 
\end{tikzcd}
\]
Then the outer square is Cartesian.

\proof

Volunteer

\qed

\subsection*{\underline{Fibre products part 2}}

\prop

Let 
\[
\begin{tikzcd}
& Z \ar[d, "g"'] \\
X \ar[r, "f"] & Y
\end{tikzcd}
\]
be a diagram of schemes. Then the fibre product $X\times_YZ$ exists in $\LRS$ and is a scheme.

\proof

\underline{Recall:} This is true if $X, Y, Z$ are affine by the universal property of $\Spec$. 

\underline{Sketch of proof:}\,
\begin{enumerate}

\item Choose open covers $(U_\alpha),(V_\alpha),(W_\alpha)$ of $X, Y, Z$ by affine schemes, so that $f:U_\alpha\to V_\alpha$ and $g:W_\alpha\to V_\alpha$ via restriction, i.e. $|f|(|U_\alpha|) \subset |V_\alpha| \supset g(|W_\alpha|)$

\item Glue $U_\alpha\times_{V_\alpha}W_\alpha$ to a scheme we can call $X\times_YZ$ 

\item Prove it satisfies the universal property. 

\end{enumerate}

\section*{Lecture 12, 1/10/25}

\underline{Recall:} Cartesian Squares $=$ fibre product squares. 


Cartesian squares can be juxtaposed as in the previous lemma.

Test on friday (next lecture) 

We have shown that fibre prodcuts along open immersions exist, and are open immersions (I denote open immersions by $\into$ when the diagram is in the category of schemes/locally ringed spaces). 

In particular, they exist in the category of schemes. 

\lem

Let 
\[
\begin{tikzcd}
& Z \ar[d] \\
X \ar[r] & Y 
\end{tikzcd}
\]
be a diagram in a category $\mc{C}$ and $Y'\to Y$ be a morphism. Assume that the fibre products $X'= X\times_Y Y $ and $Z' = Z\times_YY'$ exist, and that $X\times_YZ$ and $X'\times_{Y'}Z'$ exist. Then
\[
X'\times_{Y'}Z' \cong (X\times_YZ)\times_YY'
\]
As an exercise, complete this cube (imagine the things on the right are ``folded" up)
\[
\begin{tikzcd}
X' \ar[r] \ar[d] & Y' \ar[d] & Z'\ar[d]\ar[l] \\
X \ar[r] & Y  & Z \ar[l]
\end{tikzcd}
\]

\lem

Let $W, F \in \LRS$. Then the following presheaf, $\Hom(-,F):\operatorname{Open}(|W|)^{op}\to\Set$ is a sheaf $|U|\mapsto\Hom_{\LRS}(U,W)$, as in the commutative diagram
\[
\begin{tikzcd}
\operatorname{Open}(|W|)^{op} \ar[rr] \ar[dr] & & \Set \\
& \LRS^{op} \ar[ur,"Hom"] & \\
\end{tikzcd}
\]
Where ``$Hom$" means $\Hom(-,F)$ (tikzcd freaks out if there are parens for some reason :( ) 

\proof

Volunteer. 

\qed

Here is the main lemma of today:

\lem

Let
\[
\begin{tikzcd}
& Y \ar[d, "g"] \\
X \ar[r, "f"] & Y
\end{tikzcd}
\]
be a diagram of schemes. Assume that each scheme has an open cover $X = X_1 \cup X_2, Y = Y_1\cup Y_2, Z = Z_1 \cup Z_2$, such that $f(|X_i|) \subset |Y_i| \supset g(|Z_i|)$, and $X_i \subset X, Y_i \subset Y, Z_i \subset Z$ are open subschemes for $i \in \{1,2\}$. 

If $X_i\times_{Y_i}Z_i$ exists and is a scheme for each $i$ then $X\times_YZ$ exists and is a scheme. 

\proof

At first we need the following.

\claim

$X_{12}\times_{Y_{12}}Z_{12}$ exists and is a scheme, where $X_{12} = X_1\cap X_2$, and similarly for $Y_{12},Z_{12}$. 
\proof
Consider the fibre product
\[
\begin{tikzcd}
X_1\times_{Y_1}Z_1 \ar[d]\ar[r] & Z_1\ar[d] \\
X_1 \ar[r] & Y_1
\end{tikzcd}
\]
By the lemma we can take the fibre product 
\[
\begin{tikzcd}
\ar[d]X_{12}\times_{Y_1}Z_1 \ar[r, hook] & X_1\times_{Y_1}Z_1 \ar[d]\ar[r] & Z_1\ar[d] \\
X_{12} \ar[r, hook] & X_1 \ar[r] & Y_1
\end{tikzcd}
\]
Then we can extend this to the diagram (all $4$ squares are Cartesian)
\[
\begin{tikzcd}
X_{12}\times_{Y_1}Z_{12} \ar[d] \ar[r] & X_1\times_{Y_1}Z_{12} \ar[d, hook] \ar[r] & Z_{12} \ar[d, hook] \\
\ar[d]X_{12}\times_{Y_1}Z_1 \ar[r, hook] & X_1\times_{Y_1}Z_1 \ar[d]\ar[r] & Z_1\ar[d] \\
X_{12} \ar[r, hook] & X_1 \ar[r] & Y_1
\end{tikzcd}
\]

Then by the other lemma
\[
(X_{12}\times_{Y_{1}}Z_{12}) \times_{Y_1}Y_{12} \cong X_{12}\times_{Y_{12}}Z_{12}
\]

\qed

Now: Pushout of 
\[
\begin{tikzcd}
X_{12}\times_{Y_{12}}Z_{12} \ar[r, hook] \ar[d, hook] & X_2\times_{Y_2}Z_2\ar[d, hook] \\
X_1\times_{Y_1}Z_1 \ar[r, hook] & X\times_YZ = P
\end{tikzcd}
\]
\claim

$P$ satisfies the universal product of the fibre product. 

\proof

This follows from the sheaf property: use the lemma for an arbitrary $W$, letting $F = P$. 

Sketch:

Take the diagram
\[
\begin{tikzcd}
W \ar[drr, bend left = 20, "\beta"] \ar[ddr, bend right = 20, "\alpha"]\ar[dr, dotted, "\exists!"] \\
& P \ar[r] \ar[d] & Z \ar[d] \\
& X \ar[r] & Y 
\end{tikzcd}
\]
Apply sheaf property to an appropriate open cover $W = \cup_{i\in I}U_i$ such that $\alpha(U_i) \subset X_1\cup X_2$ and similarly for $\beta$ and $Y_1\cup Y_2$. 

\qed

This proves the main lemma

\qed

\thm

Let 
\[
\begin{tikzcd}
& Z \ar[d] \\
X \ar[r] & Y \\
\end{tikzcd}
\]
be a diagram of schemes. Then $X\times_YZ \in \LRS$ exists and is a scheme. 

\proof

Let $\mc{S}$ be the partially ordered set of open subdiagrams
\[
\begin{tikzcd}
& W \ar[dr, phantom, "\subset"] \ar[d] & \\
U \ar[r] \ar[dr, phantom, "\subset"] & V \ar[dr, phantom, "\subset"] & Z \ar[d] \\
& X \ar[r] & Y
\end{tikzcd}
\]

Then the fibre product $U\times_VW$ exists and is a scheme: 

\[
\begin{tikzcd}
U\times_VW \ar[r] \ar[dr,phantom,"\ulcorner", very near start]\ar[d] & W \ar[dr, phantom, "\subset"] \ar[d] & \\
U \ar[r] \ar[dr, phantom, "\subset"] & V \ar[dr, phantom, "\subset"] & Z \ar[d] \\
& X \ar[r] & Y
\end{tikzcd}
\]

We know that $\mc{S} \neq \varnothing$, since $X,Y,Z$ are covered by affine opens and for diagrams of affine schemes the fibre product exists and is a scheme. 

we also have that for each chain of such diagram s
\[
\begin{tikzcd}
& W_i \ar[d] & \\
U_i \ar[r] & V_i \ar[dr, phantom, "\subset"] & Z \ar[d] \\
& X\ar[r] & Y
\end{tikzcd}
\]
we can take the ascending union sheaf property to get an upper bound. 

By Zorn's lemma, there is a maximal open subdiagram 
\[
\begin{tikzcd}
& W_0 \ar[d] \\
U_0 \ar[r] & Z_0\\
\end{tikzcd}
\]
such that $U_0\times_{V_0}W_0$ exists and is a scheme. 

If $U_0\neq X$ and $V_0 \neq Y$ and $W_0\neq Z$, then we could find an open subdiagram of affine subschemes
\[
\begin{tikzcd}
& W_1 \ar[d] \\
U_1 \ar[r] & V_1
\end{tikzcd}
\]
such that $U_0 \cup U_1$ or $V_0 \cup V_1$ or $W_1 \cup W_0$ is strictly larger. Applying the main lemma above, we get a contradiction. 

\qed




\end{document}










